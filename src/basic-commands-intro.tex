\part{Commandes Basiques}

\section{Raccourcis utiles}

\begin{description}
	\item[Entrée]: le plus important, lance votre commande
	\item[Flêche haut]: naviguer dans l'historique des commandes. Vous pourrez
		ainsi gagner du temps et éviter de toujours retaper la même chose.
	\item[Tab]: touche de complétion. Vous permet de taper les quelques
		premières lettres de votre commande, puis la compléter en appuyant sur
		la touche de tabulation. Si rien ne s'affiche la première fois que vous
		appuyez sur la touche, c'est qu'il existe plusieurs possibilités de
		complétion. Une seconde pression vous permettra d'afficher toutes les
		complétions possibles.
	% TODO: move that to job control section
	% \item[Ctrl + Z]: pause la commande en cours.
	\item[Ctrl + C]: arrête la commande en cours et recommence avec une
		nouvelle ligne. Très utile si vous voulez annuler l'exécution d'un
		programme qui boucle de manière infinie, ou si vous vous rendez compte
		que vous êtes en train d'écrire n'importe quoi.
	\item[Ctrl+L]: nettoie l'écran.
	\item[Ctrl+Shift+C]: Copie le texte sélectionné dans le presse-papier.
	\item[Ctrl+Shift+V]: Colle le texte contenu dans le presse-papier.
	\item[Ctrl+Z]: Mets la commande en pause. On expliquera plus tard le
		fonctionnement. Au cas où vous appuyez sur \textbf{Ctrl+Z} par
		inadvertance, sachez que pour la reprendre, il faut taper la commande
		\texttt{fg}.
\end{description}

\section{Liste de commandes (formulaire)}

\paragraph{} Voici donc la liste des commandes qui seront abordées plus en
détail par la suite.

\begin{description}
\item[man]: \emph{reference \textbf{man}uals} affiche un manuel sur une
	commande, une fonction, ou une bibliothèque.

  \begin{lstlisting}
# Manuel d'utilisation de la command man
man man
  \end{lstlisting}

\item[cat]: \emph{con\textbf{cat}enate files} affiche un ou plusieurs fichiers
	sur le terminal.

  \begin{lstlisting}
cat plop.txt
  \end{lstlisting}

\item[ls]: \emph{\textbf{l}i\textbf{s}t directory contents} permet d'afficher
	le contenu d'un répertoire.

  \begin{lstlisting}
ls
  \end{lstlisting}

\item[cd]: \emph{\textbf{c}hange \textbf{d}irectory} permet de naviguer à
	travers les répertoires.

  \begin{lstlisting}
cd /usr/src/linux
  \end{lstlisting}

\item[cp]: \emph{\textbf{c}o\textbf{p}y files} copie un fichier ou un dossier.

  \begin{lstlisting}
# Pour un fichier
cp /tmp/plop.txt /home/nom-d-utilisateur/
  \end{lstlisting}
%# Pour un  répertoire
%cp -r /tmp/foo /home/nom-d-utilisateur
  %\end{lstlisting}

\item[mv]: \emph{\textbf{m}o\textbf{v}e file} déplace un fichier ou un dossier.

  \begin{lstlisting}
mv /tmp/foo /home/nom-d-utilisateur
  \end{lstlisting}

\item[./<exec>]: lance le script/programme \emph{<exec>} qui se trouve dans le
	dossier actuel.

  \begin{lstlisting}
# Pour lancer a.out
./a.out
  \end{lstlisting}

\end{description}

\section{Détails des commandes}
\subsection{man}

\subsubsection*{Description}

\paragraph{} \emph{man} sert à consulter le manuel du système. Il s'agit là
probablement de la commande la plus utile pour un débutant dans un terminal.
Elle prend en paramètre une ou plusieurs commandes dont on veut connaître la
description et l'utilisation.

\subsubsection*{Exemple d'utilisation}

\begin{lstlisting}
$ man cal
CAL(1)                   User Commands                   CAL(1)



NAME
       cal - display a calendar

SYNOPSIS
       cal [options] [[[day] month] year]

DESCRIPTION
       cal  displays  a  simple  calendar.  If no arguments are
       specified, the current month is displayed.

OPTIONS
       -1, --one
              Display  single  month  output.   (This  is   the
              default.)

       -3, --three
              Display three months spanning the date.

       -n , --months number
              Display number of months, starting from the month
              containing the date.
...
\end{lstlisting}

\paragraph{} Parmi les informations importantes, on peut voir le
\emph{synopsis} de la commande. Les mots entre crochets sont facultatifs. Cela
signifie donc que toutes les options (commençant par des tirets) sont
facultatives, et que spécifier l'année est facultatif, de même que le mois si
l'année est présente, etc\ldots

\paragraph{} Par exemple, on peut écrire comme commandes:
\begin{itemize}
	\item \texttt{cal}, ce qui donne:
\begin{lstlisting}
   septembre 2015
lu ma me je ve sa di
    1  2  3  4  5  6
 7  8  9 10 11 12 13
14 15 16 17 18 19 20
21 22 23 24 25 26 27
28 29 30
\end{lstlisting}
	\item \texttt{cal -1}, qui nous donne le même résultat que la commande
		précédente
	\item \texttt{cal 2015}, qui nous donne le calendrier pour 2015
		(un peu long).
	\item \texttt{cal ----months 4 02 2014} (affiche 4 mois à partir de février
		2014), nous donnant donc:
\begin{lstlisting}[basicstyle=\footnotesize\ttfamily]
    février 2014            mars 2014            avril 2014
lu ma me je ve sa di  lu ma me je ve sa di  lu ma me je ve sa di
                1  2                  1  2      1  2  3  4  5  6
 3  4  5  6  7  8  9   3  4  5  6  7  8  9   7  8  9 10 11 12 13
10 11 12 13 14 15 16  10 11 12 13 14 15 16  14 15 16 17 18 19 20
17 18 19 20 21 22 23  17 18 19 20 21 22 23  21 22 23 24 25 26 27
24 25 26 27 28        24 25 26 27 28 29 30  28 29 30
                      31
      mai 2014
lu ma me je ve sa di
          1  2  3  4
 5  6  7  8  9 10 11
12 13 14 15 16 17 18
19 20 21 22 23 24 25
26 27 28 29 30 31
\end{lstlisting}
\end{itemize}

\subsection{cat}
\subsubsection*{Description}

\paragraph{} \emph{cat} est une commande permettant d'afficher le contenu d'un
ou plusieurs fichiers dans le terminal. Elle prend donc comme paramètre un, ou
une liste de fichiers: texte, code source, script\ldots pour l'afficher
directement sur le terminal.

\subsubsection*{Exemple d'utilisation}

\begin{lstlisting}
$ ls
texte.txt code.c image.png
$ cat texte.txt
Je suis le contenu du fichier texte!
$ cat code.c texte.txt
#include <stdio.h>

int main(){
	printf("Hello, World\n");
	return 0;
}
Je suis le contenu du fichier texte!
\end{lstlisting}

\paragraph{} Attention toutefois: lors de son utilisation sur un fichier
binaire (qui ne contient pas du texte classique), la commande s'exécutera bien.
Cependant, le contenu du fichier est souvent trop gros pour votre terminal et
inintelligible, et peut casser l'affichage de votre terminal. Cela signifie par
exemple que lorsque vous tapez des caractères, des symboles étranges
apparaîtront.

\paragraph{} Si par inadvertance cela vous arrivait, vous pourrez toujours
essayer de taper "\emph{reset}" pour rétablir votre shell (il n'y a pas
d'inquiétude à avoir si vous ne voyez pas la commande "\emph{reset}"
s'afficher, les caractères tapés seront bien pris en compte).

\subsection{ls}
\subsubsection*{Description}

\paragraph{} \texttt{ls} liste les fichiers et dossiers présents dans le
répertoire courant. Il prend en paramètre un dossier dont on veut lister les
fichiers. Sans paramètre cette commande liste les fichiers contenus dans le
répertoire courant. Il existe plusieurs options utiles à cette commande, comme
\texttt{-l} qui donne des informations détaillées sur les fichiers. Une autre
option utile est \texttt{----color} qui permet d'obtenir des indications sur
les fichiers en colorant leurs noms; les fichiers exécutables
deviennent vert, les dossiers bleus\ldots

\subsubsection*{Exemple d'utilisation}
\begin{lstlisting}
$ ls
Dossier/
$ ls Dossier/
texte.txt code.c image.png
$ ls -l Dossier/
total 20
-rw-r--r-- 1 club user    13 déc.   6 01:32 texte.txt
-rwxr-xr-x 1 club user 10019 déc.   6 01:33 image.png
-rw-r--r-- 1 club user   301 déc.   6 01:33 code.c
\end{lstlisting}

\paragraph{} On voit que l'option \texttt{-l} affiche plein d'informations
utiles. La première colonne indique les droits sur le fichier, que l'on
expliquera après, la troisième colonne avec \texttt{club} indique le
propriétaire du fichier (donc le propriétaire est l'utilisateur \texttt{club}).
La quatrième colonne représente le groupe propriétaire du fichier. Dans cet
exemple, le groupe propriétaire est le groupe \texttt{user}, qui est le groupe
des utilisateurs normaux. La cinquième colonne est la taille du fichier en
octets. Attention, la taille affichée par \texttt{ls -l} pour les dossiers ne
correspond pas à la taille de tous les sous-fichiers, et sous-dossiers. La
sixième colonne est la date de dernière modification du fichier (à savoir ici
le 6 décembre 1h33).

\paragraph{} Pour la première colonne, il faut savoir que les fichiers sous
GNU/Linux ont trois permissions principales: le droit de lecture (représenté
par \texttt{r} pour \emph{read}), le droit d'écriture (représenté par
\texttt{w} pour \emph{write}) et le droit d'exécution (représenté par
\texttt{x} pour \emph{eXecute}).

\paragraph{} La première lettre de cette colonne correspond au type de fichier.
Si par exemple, il s'agit d'un dossier, on verra la lettre \texttt{d}.  Et si
c'est un fichier normal, ce sera un \texttt{-}. Les trois prochaines lettres
correspondent respectivement au droit de lecture, écriture et exécution pour le
propriétaire du fichier: si la lettre (\texttt{r}, \texttt{w} ou \texttt{x})
est présente, cela veut dire que le propriétaire possède le droit, sinon, la
lettre sera remplacée par un \texttt{-}. Les trois lettres suivantes sont pour
les membres du groupe auquel appartient le fichier, et les trois dernières
lettres sont pour les autres utilisateurs.

\paragraph{} Par exemple, si l'on a \texttt{-rw-r----r----}, cela veut dire
qu'il s'agit d'un fichier normal et que le propriétaire peut lire le fichier,
le modifier (et donc le supprimer) mais pas l'exécuter. Les membres du même
groupe et les autres utilisateurs peuvent seulement le lire. Autre exemple: si
l'on a \texttt{drwxr-xr-x}, on a là un dossier qui peut être lu (donc voir son
contenu), écrit (rajouter des fichiers dedans) et exécuté (pouvoir ``rentrer''
dans le dossier) par le propriétaire. Les membres du groupe et les autres
utilisateurs peuvent seulement le lire et l'exécuter.

\subsection{cd}
\subsubsection*{Description}

\paragraph{} \texttt{cd} permet de naviguer à travers l'arborescence des
dossiers. Comme paramètre, cette commande accepte aussi bien un chemin relatif
qu'absolu. Sans paramètre, cette commande vous mènera à votre dossier personnel
(\texttt{\~}).

\subsubsection*{Exemples d'utilisation}

\begin{lstlisting}
~/$ cd Documents
~/Documents/$ cd ESIEE/man-terminal
~/Documents/ESIEE/man-terminal/$ cd
~/$
\end{lstlisting}

\subsection{cp}

\subsubsection*{Description}

\paragraph{} \texttt{cp} est une commande permettant de copier des fichiers.
Cette commande prend en paramètres deux noms de fichier. Le premier est le
fichier \emph{source}, fichier qui doit être copié. Le second est la
destination : il s'agit soit d'un dossier, auquel cas le nom de fichier sera
préservé, soit d'un chemin vers un fichier (qui sera écrasé si il existe
déjà).

\paragraph{} Afin de copier tout un dossier, il faut copier les fichiers et les
sous-dossiers qui sont dedans, c'est à dire copier de manière \emph{récursive}.
On utilise alors l'option \texttt{-r} de \texttt{cp}. Sans ce modificateur, la
commande cp omettra la copie de dossier.

\subsubsection*{Exemples d'utilisation}

\begin{lstlisting}[caption=copie de fichier]
~/$ ls
texte.txt code.c image.png
~/$ cp texte.txt texte2.txt
~/$ ls
texte.txt texte2.txt code.c image.png
~/$
\end{lstlisting}

\begin{lstlisting}[language=bash,caption=copie de dossier]
~/$ ls
Dossier/
~/$ ls Dossier/
texte.txt code.c image.png
~/$ cp -r Dossier/ Dossier2/
~/$ ls
Dossier/ Dossier2/
~/$ ls Dossier2/
texte.txt code.c image.png
~/$
\end{lstlisting}

\subsection{mv}

\subsubsection*{Description}

\paragraph{} \texttt{mv} sert à déplacer des fichiers, il peut aussi déplacer
tout un dossier sans avoir besoin de rajouter d'options. Lors du déplacement,
on peut aussi donner un nouveau nom au fichier. Ainsi pour renommer un fichier,
on le déplace au même endroit. Son utilisation est similaire à la commande
\texttt{cp} (hormis le \texttt{-r} qui n'est pas nécéssaire).

\subsubsection*{Exemple d'utilisation}

\begin{lstlisting}[caption=déplacement d'un fichier]
$ ls
texte.txt code.c image.png Sous-Dossier/
$ ls Sous-Dossier/

$ mv texte.txt Sous-Dossier/
$ ls Chaton/
texte.txt
$ ls
code.c image.png Sous-Dossier/
\end{lstlisting}

\begin{lstlisting}[caption=renommer un fichier]
$ ls
texte.txt code.c image.png Sous-Dossier/
$ mv texte.txt citation.txt
$ ls
citation.txt code.c image.png Sous-Dossier/
\end{lstlisting}

\subsection{./exec}

\subsubsection*{Description}

\paragraph{} Afin d'exécuter un programme, le shell doit comprendre que le
programme se situe dans un dossier et non dans la liste des commandes
installées.  Pour ce faire, on peut lui donner le chemin absolu ou relatif vers
le logiciel. Taper directement \emph{exec} sera cependant considéré comme une
commande à chercher dans une destination standard (là où se trouvent les autres
commandes \texttt{cat}, \texttt{ls}\ldots).  On utilise donc le fichier spécial
\emph{.}, représentant le dossier courant, pour finalement obtenir
\emph{./exec}, signifiant donc \emph{"l'exécutable exec se situant dans le
	dossier courant"}.

\subsubsection*{Exemple d'utilisation}
\begin{lstlisting}
$ ./monExecutable
Hello, World!
$ 
\end{lstlisting}


\vspace{7em}
