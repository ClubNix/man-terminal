\part{Commandes Basiques}

\section{Raccourcis utiles}

\begin{description}
	\item[Entrée]: le plus important, lance votre commande
	\item[Flêche haut]: naviguer dans l'historique des commandes. Vous pourrez
		ainsi gagner du temps et éviter de toujours retaper la même chose.
	\item[Tab]: touche de complétion. Vous permet de taper les quelques
		premières lettres de votre commande, puis la compléter en appuyant sur
		la touche de tabulation. Si rien ne s'affiche la première fois que vous
		appuyez sur la touche, c'est qu'il existe plusieurs possibilités de
		complétion. Une seconde pression vous permettra d'afficher toutes les
		complétions possibles.
	% TODO: move that to job control section
	% \item[Ctrl + Z]: pause la commande en cours.
	\item[Ctrl + C]: arrête la commande en cours et recommence avec une
		nouvelle ligne. Très utile si vous voulez annuler l'exécution d'un
		programme qui boucle de manière infinie, ou si vous vous rendez compte
		que vous êtes en train d'écrire n'importe quoi.
	\item[Ctrl+L]: nettoie l'écran.
	\item[Ctrl+Shift+C]: Copie le texte sélectionné dans le presse-papier.
	\item[Ctrl+Shift+V]: Colle le texte contenu dans le presse-papier.
	\item[Ctrl+Z]: Mets la commande en pause. On expliquera plus tard le
		fonctionnement. Au cas où vous appuyez sur \textbf{Ctrl+Z} par
		inadvertance, sachez que pour la reprendre, il faut taper la commande
		\texttt{fg}.
\end{description}

\section{Liste de commandes (formulaire)}

\paragraph{} Voici donc la liste des commandes qui seront abordées plus en
détail par la suite.

\begin{description}
\item[man]: \emph{reference \textbf{man}uals} affiche un manuel sur une
	commande, une fonction, ou une bibliothèque.

  \begin{lstlisting}
# Manuel d'utilisation de la command man
man man
  \end{lstlisting}

\item[cat]: \emph{con\textbf{cat}enate files} affiche un ou plusieurs fichiers
	sur le terminal.

  \begin{lstlisting}
cat plop.txt
  \end{lstlisting}

\item[ls]: \emph{\textbf{l}i\textbf{s}t directory contents} permet d'afficher
	le contenu d'un répertoire.

  \begin{lstlisting}
ls
  \end{lstlisting}

\item[cd]: \emph{\textbf{c}hange \textbf{d}irectory} permet de naviguer à
	travers les répertoires.

  \begin{lstlisting}
cd /usr/src/linux
  \end{lstlisting}

\item[cp]: \emph{\textbf{c}o\textbf{p}y files} copie un fichier ou un dossier.

  \begin{lstlisting}
# Pour un fichier
cp /tmp/plop.txt /home/nom-d-utilisateur/
  \end{lstlisting}
%# Pour un  répertoire
%cp -r /tmp/foo /home/nom-d-utilisateur
  %\end{lstlisting}

\item[mv]: \emph{\textbf{m}o\textbf{v}e file} déplace un fichier ou un dossier.

  \begin{lstlisting}
mv /tmp/foo /home/nom-d-utilisateur
  \end{lstlisting}

\item[./<exec>]: lance le script/programme \emph{<exec>} qui se trouve dans le
	dossier actuel.

  \begin{lstlisting}
# Pour lancer a.out
./a.out
  \end{lstlisting}

\end{description}

\section{Détails des commandes}
\subsection*{man}
\subsubsection*{Description}
\emph{man} sert à consulter le manuel du système. 

\subsubsection*{Exemple d'utilisation}

\begin{lstlisting}
$ man man
$ man cat
$ man rsync
\end{lstlisting}

\subsection{Cat}
\subsubsection{Description}
\emph{Cat} est une commande permettant d'afficher du texte sur l'entrée standard.
Elle prend en entrée un fichier: texte, code source, script... pour l'afficher directement sur le terminal.
Attention toutefois, en l'utilisant sur un fichier binaire, la commande s'exécutera et en essayant d'afficher le contenu du fichier souvent trop gros pour votre terminal, et un fichier binaire correspondant a une suite aléatoire de caractère ascii, cassera l'affichage de celui ci.
Si par inadvertance cela vous arrivait, vous pouvez toujours essayer de taper "reset" pour rétablir votre shell (pas d'inquiétude si vous ne voyez rien, la commande est quand même écrite correctement).

\subsubsection{Exemple d'utilisation}

\begin{lstlisting}
$ls
chaton.txt chat.c poulpe.exe
$cat chat.c
#include <stdio.h>

int main(){
	printf("oh, un çhat!");
 return 0;
}
$
\end{lstlisting}

\subsection*{ls}
\subsubsection*{Description}
\emph{ls} liste les fichiers et dossiers présents dans le répertoire courant.
Il existe plusieurs options utiles à cette commande comme \emph{-l} qui donne des informations détaillées sur les fichiers.
Une autre option utile est \emph{-{}-color} qui permet d'obtenir des indications sur les fichiers en colorant les noms de ceux-ci, les fichiers exécutables deviennent vert, les dossiers bleus ...
Pour ceux qui se souviennent encore de \emph{DOS}, cette commande est l'equivalent (quoique beaucoup plus complète) de \emph{dir}, il en existe d'ailleurs un alias sur les ordinateurs à l'esiee \textcolor{grey2}{à vérifier}

\subsubsection*{Exemple d'utilisation}
\begin{lstlisting}
$ ls
Chaton/
$ ls Chaton/
miaou.txt chat.c poulpe.exe
$ ls -l Chaton/
total 20
-rw-r--r-- 1 club user    13 déc.   6 01:32 miaou.txt
-rwxr-xr-x 1 club user 10019 déc.   6 01:33 pouple.exe
-rw-r--r-- 1 club user   301 déc.   6 01:33 value.c
\end{lstlisting}

\subsection*{cd}
\subsubsection*{Description}
\emph{cd} permet de naviguer à travers l'arborescence des dossiers.
En plus de pouvoir se déplacer dans un fichier, cette commande accepte plusieurs caractères spéciaux.
En essayant de se déplacer dans le dossier "..", la commande permet de revenir au dossier précédent.
Il est aussi possible de revenir directement à la racine de ces fichiers, sa "home" à l'aide du caractère "\~", à noter qu'omettre un nom de dossier a la même conséquence.

La commande cd accepte des chemins \emph{absolus}, "/home/nepta/Poly\textbackslash Linux/", mais aussi des chemins \emph{relatifs}, "Poly\textbackslash \textvisiblespace Linux" permettant de se déplacer en fonction du dossier courant.

Certains caractères, notamment le caractère espace (" ") doivent être \emph{échappés} afin d'être pris en compte, pour cela le caractère "\textbackslash" est utilisé ("Poly Linux" devient "Polyt\textbackslash Linux",
"Chaton\textbackslash Chat\textbackslash Tigre" devient "Chaton\textbackslash \textbackslash Chat\textbackslash \textbackslash Tigre")

\subsubsection*{Exemples d'utilisations}

\begin{lstlisting}
~/$ cd Poly\ Linux
~/Poly Linux/$ cd Chaton/Petit\ Chat/
~/Poly Linux/Chaton/Petit Chat/$ cd
~/ $ 
\end{lstlisting}

\subsection{cp}
\subsubsection{Description}
\emph{cp} est une commande permettant de copier des fichier.
Cette commande prend en paramètres deux nom de fichier, le premier est le fichier \emph{source}, fichier qui doit être copié, le second, la destination qui est soit un dossier auquel cas le nom de fichier sera préservé, soit un chemin vers un fichier (qui sera écraser si il existe déjà).\\
Afin de copier tout un dossier, le mode récursif de \emph{cp} est utilisé, pour cela l'on ajoute le modificateur de commande "-r" qui permettra la copie de tous les répertoire, sous répertoire et fichier contenu depuis la source jusqu'à la destination.
Sans se modificateur, la commande cp ometra la copie de dossier.

\subsubsection{Exemples d'utilisations}

\begin{lstlisting}[caption=copie de fichier]
~/$ ls
miaou.txt chat.c poulpe.exe
~/$ cp miaou.txt chaton.txt
~/$ ls
miaou.txt chaton.txt chat.c poulpe.exe
~/$
\end{lstlisting}

\begin{lstlisting}[language=bash,caption=copie de dossier]
~/$ ls
Chaton/
~/$ ls Chaton/
miaou.txt chat.c poulpe.exe
~/$ cp -r Chaton/ Miaou/
~/$ ls
Chaton/ Miaou/
~/$ ls Miaou/
miaou.txt chat.c poulpe.exe
~/$
\end{lstlisting}

\subsection{mv}

\subsubsection*{Description}

\paragraph{} \texttt{mv} sert à déplacer des fichiers, il peut aussi déplacer
tout un dossier sans avoir besoin de rajouter d'options. Lors du déplacement,
on peut aussi donner un nouveau nom au fichier. Ainsi pour renommer un fichier,
on le déplace au même endroit. Son utilisation est similaire à la commande
\texttt{cp} (hormis le \texttt{-r} qui n'est pas nécéssaire).

\subsubsection*{Exemple d'utilisation}

\begin{lstlisting}[caption=déplacement d'un fichier]
$ ls
texte.txt code.c image.png Sous-Dossier/
$ ls Sous-Dossier/

$ mv texte.txt Sous-Dossier/
$ ls Chaton/
texte.txt
$ ls
code.c image.png Sous-Dossier/
\end{lstlisting}

\begin{lstlisting}[caption=renommer un fichier]
$ ls
texte.txt code.c image.png Sous-Dossier/
$ mv texte.txt citation.txt
$ ls
citation.txt code.c image.png Sous-Dossier/
\end{lstlisting}

\subsection{./exec}

\subsubsection*{Description}

\paragraph{}
Afin d'executer un programme, l'interpréteur de commande doit comprendre que le
programme se situe dans un dossier et non dans la liste des commandes
installées.  Pour ce faire on peut lui donner le chemin absolu vers le logiciel
ou relatif, taper directement \emph{exec} sera cependant considéré comme une
commande à chercher dans une destination standard (là où se trouve les autres
commandes cat,ls ...).  On utilise donc le fichier special \emph{.},
représentant le dossier courant, pour finalement obtenir \emph{./exec}
signifiant donc \emph{"l'exécutable exec se situant dans le dossier courant"}.

\subsubsection*{Exemple d'utilisation}
\begin{lstlisting}
$ ./poulpe.exe
oh, un çhat!
$ 
\end{lstlisting}


\vspace{7em}
