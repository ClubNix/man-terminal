\part{Chemins (naviguer dans les dossiers)}

\paragraph{} Un chemin, ou chemin d'accès (\emph{path} en anglais) représente
tout simplement la position d'un fichier ou d'un dossier dans le système. Sous
GNU/Linux, les chemins sont séparés par des barres obliques orientés vers la
droite: ``\texttt{/}''.  Cela explique le fait que ce caractère soit interdit
dans les noms de fichiers et dossiers sous GNU/Linux.

\paragraph{Exemples:}

\begin{itemize}
	\item \texttt{/home/nom-d-utilisateur/Documents/Bla.pdf}
	\item \texttt{Musique/thing.mp3}
	\item \texttt{\~{}/Images/wallpaper.png}
	\item \texttt{../man-terminal.pdf}
\end{itemize}

\section{Chemins absolus}

\paragraph{} Les chemins absolus sont des chemins qui prennent comme point de
départ la racine du système (\emph{root} en anglais) dont tout les fichiers et
dossiers sont des descendants. Un chemin relatif commencera toujours par un
\emph{slash} (``\texttt{/}'').

\paragraph{} Ainsi, \texttt{/home/nom-d-utilisateur/Documents/Bla.pdf}
représente le fichier \texttt{Bla.pdf} qui est situé dans le dossier
\texttt{Documents}, lui-même étant dans le dossier \texttt{nom-d-utilisateur},
lui-même contenu dans le dossier \texttt{home} qui est un descendant direct du
dossier racine. Cela donne donc (en omettant les autres dossiers et fichiers):
\\
\dirtree{%
	.1 / (root).
		.2 home.
			.3 nom-d-utilisateur.
				.4 Documents.
					.5 Bla.pdf.
}

\section{Chemins relatifs}

\paragraph{} Les chemins relatifs, à la différence des chemins absolus,
prennent en compte le dossiers dans lequel on est. Ainsi, un chemin relatif ne
mènera pas au même fichier selon où l'on est. Ces types de chemins sont très
utiles pour les fainéants qui ne veulent pas tout re-taper depuis le dossier
racine. Les chemins relatifs ne commencent pas par un \emph{slash}.

\paragraph{} Par exemple, \texttt{Musique/thing.mp3} représente le fichier
\texttt{thing.mp3} qui est dans le dossier \texttt{Musique} lui même qui est
dans le dossier actuel. En supposant que le dossier actuel soit le dossier
personnel (\texttt{/home/nom-d-utilisateur}), cela nous donne:
\\
\dirtree{%
	.1 / (root).
		.2 home.
			.3 nom-d-utilisateur (dossier actuel).
				.4 Musique.
					.5 thing.mp3.
}

\section{Chemins spéciaux}

\paragraph{} Parmis les chemins sous GNU/Linux, il y a quelques chemins
spéciaux:

\begin{itemize}
	\item ``\texttt{\~}'' correspond au dossier personnel
		(\texttt{/home/nom-d-utilisateur}), ainsi
	\item ``\texttt{.}'' correspond au dossier actuel\\
	\item ``\texttt{..}'' correspond au dossier parent\\
\end{itemize}

\paragraph{Exemples:}~\\

\begin{tabular}{|l|l|}
	\hline
	\textbf{Chemin} & \textbf{Équivalence}\\
	\hline
	\texttt{$\sim$/Musique} & \texttt{/home/nom-d-utilisateur/Musique}\\
	\hline
	\texttt{Documents/./texte.txt} & \texttt{Documents/texte.txt}\\
	\hline
	\texttt{Documents/../Téléchargements/cat.gif} & \texttt{Téléchargements/cat.gif}\\
	\hline
\end{tabular}

\section{Caractères spéciaux}

\paragraph{} Sous GNU/Linux, le seul caractère présent sur le clavier non
autorisé dans les noms de fichiers et dossiers est le \emph{slash}. Cependant
il existe d'autres caractères spéciaux pour le shell qui devront être
\emph{échappés}, ce qui veut dire qu'il devront être précédés d'un
\emph{antislash} (``\texttt{\textbackslash}''). Les plus utilisés sont:

\begin{itemize}
	\item \textvisiblespace~(l'espace)
	\item \texttt{"}
	\item \texttt{'}
	\item \texttt{!}
	\item \texttt{\#}
	\item \texttt{\$}
	\item \texttt{(}, \texttt{)}, \texttt{[}, \texttt{]}, \texttt{\{}, \texttt{\}}
\end{itemize}

\paragraph{} Donc si un fichier s'appelle \texttt{Un nom trè\$ embêtant (mais
peu probable)}, on devra écrire dans le \emph{prompt}:
\texttt{Un\textbackslash{\textvisiblespace}nom\textbackslash{\textvisiblespace}trè\textbackslash\$\textbackslash\textvisiblespace
	embêtant\textbackslash{\textvisiblespace}\textbackslash(mais\textbackslash{\textvisiblespace}peu\textbackslash\textvisiblespace
probable\textbackslash)}.

\paragraph{} Un moyen d'éviter ceci est de mettre des guillemets (simples ou
doubles) autour. Prenez garde tout de même car le symbole du dollar doit
quand-même être échappé si l'on met des guillemets doubles. Pour notre exemple,
cela donne:
\\
\begin{itemize}
	\item \texttt{"Un nom trè\textbackslash\$ embêtant (mais peu probable)"}\\
	ou
	\item \texttt{'Un nom trè\$ embêtant (mais peu probable)'}
\end{itemize}

\newpage
