\subsection{ls}
\subsubsection*{Description}

\paragraph{} \texttt{ls} liste les fichiers et dossiers présents dans le
répertoire courant. Il prend en paramètre un dossier dont on veut lister les
fichiers. Sans paramètre cette commande liste les fichiers contenus dans le
répertoire courant. Il existe plusieurs options utiles à cette commande, comme
\texttt{-l} qui donne des informations détaillées sur les fichiers. Une autre
option utile est \texttt{----color} qui permet d'obtenir des indications sur
les fichiers en colorant leurs noms; les fichiers exécutables
deviennent vert, les dossiers bleus\ldots

\subsubsection*{Exemple d'utilisation}
\begin{lstlisting}
$ ls
Dossier/
$ ls Dossier/
texte.txt code.c image.png
$ ls -l Dossier/
total 20
-rw-r--r-- 1 club user    13 déc.   6 01:32 texte.txt
-rwxr-xr-x 1 club user 10019 déc.   6 01:33 image.png
-rw-r--r-- 1 club user   301 déc.   6 01:33 code.c
\end{lstlisting}

\paragraph{} On voit que l'option \texttt{-l} affiche plein d'informations
utiles. La première colonne indique les droits sur le fichier, que l'on
expliquera après, la troisième colonne avec \texttt{club} indique le
propriétaire du fichier (donc le propriétaire est l'utilisateur \texttt{club}).
La quatrième colonne représente le groupe propriétaire du fichier. Dans cet
exemple, le groupe propriétaire est le groupe \texttt{user}, qui est le groupe
des utilisateurs normaux. La cinquième colonne est la taille du fichier en
octets. Attention, la taille affichée par \texttt{ls -l} pour les dossiers ne
correspond pas à la taille de tous les sous-fichiers, et sous-dossiers. La
sixième colonne est la date de dernière modification du fichier (à savoir ici
le 6 décembre 1h33).

\paragraph{} Pour la première colonne, il faut savoir que les fichiers sous
GNU/Linux ont trois permissions principales: le droit de lecture (représenté
par \texttt{r} pour \emph{read}), le droit d'écriture (représenté par
\texttt{w} pour \emph{write}) et le droit d'exécution (représenté par
\texttt{x} pour \emph{eXecute}).

\paragraph{} La première lettre de cette colonne correspond au type de fichier.
Si par exemple, il s'agit d'un dossier, on verra la lettre \texttt{d}.  Et si
c'est un fichier normal, ce sera un \texttt{-}. Les trois prochaines lettres
correspondent respectivement au droit de lecture, écriture et exécution pour le
propriétaire du fichier: si la lettre (\texttt{r}, \texttt{w} ou \texttt{x})
est présente, cela veut dire que le propriétaire possède le droit, sinon, la
lettre sera remplacée par un \texttt{-}. Les trois lettres suivantes sont pour
les membres du groupe auquel appartient le fichier, et les trois dernières
lettres sont pour les autres utilisateurs.

\paragraph{} Par exemple, si l'on a \texttt{-rw-r----r----}, cela veut dire
qu'il s'agit d'un fichier normal et que le propriétaire peut lire le fichier,
le modifier (et donc le supprimer) mais pas l'exécuter. Les membres du même
groupe et les autres utilisateurs peuvent seulement le lire. Autre exemple: si
l'on a \texttt{drwxr-xr-x}, on a là un dossier qui peut être lu (donc voir son
contenu), écrit (rajouter des fichiers dedans) et exécuté (pouvoir ``rentrer''
dans le dossier) par le propriétaire. Les membres du groupe et les autres
utilisateurs peuvent seulement le lire et l'exécuter.
