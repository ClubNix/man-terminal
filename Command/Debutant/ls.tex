\subsection{ls}
\subsubsection{Description}
\emph{ls} liste les fichiers et dossiers présents dans le répertoire courant.
Il existe plusieurs options utile à cette commande comme \emph{-l} qui donne des informations détaillé sur les fichiers.
Une autre option utile est \emph{-{}-color} qui permet d'obtenir des indication sur les fichiers en colorant les noms de ceux-ci, les fichiers exécutable deviennent vert, les dossiers bleus ...
Pour ceux qui se souviennent encore de \emph{DOS} cette commande est l'equivalent (quoique beaucoup plus complète) de \emph{dir}, il en existe d'ailleurs un alias sur les ordinateur à l'esiee \textcolor{grey2}{à vérifier}

\subsubsection{Exemple d'utilisation}
\begin{lstlisting}
$ ls
Chaton/
$ ls Chaton/
miaou.txt chat.c poulpe.exe
$ ls -l Chaton/
total 20
-rw-r--r-- 1 club user    13 déc.   6 01:32 miaou.txt
-rwxr-xr-x 1 club user 10019 déc.   6 01:33 pouple.exe
-rw-r--r-- 1 club user   301 déc.   6 01:33 value.c
\end{lstlisting}
