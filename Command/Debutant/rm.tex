\subsection{rm}
\subsubsection*{Description}

\paragraph{} \texttt{rm} est une commande permettant de supprimer un fichier ou
un dossier. Chaque paramètre de cette commande sera donc un fichier ou dossier
à supprimer. Comme pour \texttt{cp}, l'option \texttt{-r} permet de supprimer un
dossier ainsi que tous les fichiers et sous-dossiers inclus dedans.

\subsection*{Exemple d'utilisation}

\begin{lstlisting}
$ ls
test.txt texte.txt code.c image.png Sous-Dossier/
$ rm test.txt
$ ls
texte.txt code.c image.png Sous-Dossier/
$ rm Sous-Dossier
rm: impossible de supprimer "test": est un dossier
$ rm -r Sous-Dossier
$ ls
texte.txt code.c image.png
\end{lstlisting}

\paragraph{Attention:} la commande \texttt{rm} ne va pas vous prevenir des
fichiers qu'il va supprimer ni ne va pas les placer dans la corbeille. Une fois
supprimés ces fichiers ne sont pas récupérables. Si vous voulez que \texttt{rm}
vous préviens et vous demande avant de supprimer les fichiers passé en
arguments, vous pouvez utiliser l'option \texttt{-i} pour
\emph{\textbf{i}nteractive}.
