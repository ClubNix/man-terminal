\subsection{Cat}
\subsubsection{Description}
\emph{Cat} est une commande permettant d'afficher du texte sur l'entrée standard.
Elle prend en entrée un fichier: texte, code source, script... pour l'afficher directement sur le terminal.
Attention toutefois, en l'utilisant sur un fichier binaire, la commande s'exécutera et en essayant d'afficher le contenu du fichier souvent trop gros pour votre terminal, et un fichier binaire correspondant a une suite aléatoire de caractère ascii, cassera l'affichage de celui ci.
Si par inadvertance cela vous arrivait, vous pouvez toujours essayer de taper "reset" pour rétablir votre shell (pas d'inquiétude si vous ne voyez rien, la commande est quand même écrite correctement).

\subsubsection{Exemple d'utilisation}

\begin{lstlisting}
$ls
chaton.txt chat.c poulpe.exe
$cat chat.c
#include <stdio.h>

int main(){
	printf("oh, un çhat!");
 return 0;
}
$
\end{lstlisting}
