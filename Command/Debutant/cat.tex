\subsection{cat}
\subsubsection*{Description}

\paragraph{} \texttt{cat} est une commande permettant d'afficher le contenu d'un
ou plusieurs fichiers dans le terminal. Elle prend donc comme paramètre un, ou
une liste de fichiers: texte, code source, script\ldots pour l'afficher
directement sur le terminal.

\subsubsection*{Exemple d'utilisation}

\begin{lstlisting}
$ ls
texte.txt code.c image.png
$ cat texte.txt
Je suis le contenu du fichier texte!
$ cat code.c texte.txt
#include <stdio.h>

int main(){
	printf("Hello, World\n");
	return 0;
}
Je suis le contenu du fichier texte!
\end{lstlisting}

\paragraph{} Attention toutefois: lors de son utilisation sur un fichier
binaire (qui ne contient pas du texte classique), la commande s'exécutera bien.
Cependant, le contenu du fichier est souvent trop gros pour votre terminal et
inintelligible, et peut casser l'affichage de votre terminal. Cela signifie par
exemple que lorsque vous tapez des caractères, des symboles étranges
apparaîtront.

\paragraph{} Si par inadvertance cela vous arrivait, vous pourrez toujours
essayer de taper ``\texttt{reset}'' pour rétablir votre shell (il n'y a pas
d'inquiétude à avoir si vous ne voyez pas la commande ``\texttt{reset}''
s'afficher, les caractères tapés seront bien pris en compte).
