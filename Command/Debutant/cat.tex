\subsection*{cat}
\subsubsection*{Description}
\emph{cat} est une commande permettant d'afficher du texte sur l'entrée standard.
Elle prend en entrée un fichier: texte, code source, script... pour l'afficher directement sur le terminal.

Attention toutefois : lors de son utilisation sur un fichier binaire, la commande s'exécutera bien. Cependant, le contenu du fichier est souvent trop gros pour votre terminal, et la suite de caracteres ASCII aléatoires à laquelle il correspond cassera votre terminal lorsque cat essaiera de l'afficher.
Si par inadvertance cela vous arrivait, vous pouvez toujours essayer de taper "reset" pour rétablir votre shell (il n'y a pas d'inquiétude à avoir si vous ne voyez pas la commande "reset" s'afficher, les caractères tapés sont bien pris en compte).


\subsubsection*{Exemple d'utilisation}

\begin{lstlisting}
$ ls
chaton.txt chat.c poulpe.exe
$ cat chat.c
#include <stdio.h>

int main(){
	printf("oh, un chat!\n");
 return 0;
}
$
\end{lstlisting}
