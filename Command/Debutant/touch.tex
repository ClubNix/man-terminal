\subsection{touch}
\subsection*{Description}

\paragraph{} À l'origine, \texttt{touch} a été créé pour mettre la date de
dernière modification d'un fichier à la date actuelle. Cependant, le fait
d'appeler cette commande avec en paramètre un fichier non-existant crée le
fichier, si bien que c'est devenu l'utilisation principale de cette commande.
On utilise donc \texttt{touch} pour créer un fichier vide.

\subsubsection*{Exemple d'utilisation}

\begin{lstlisting}
$ ls
texte.txt code.c image.png
$ touch test
$ ls
test texte.txt code.c image.png
\end{lstlisting}
