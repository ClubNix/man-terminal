\subsection*{cp}
\subsubsection*{Description}
\emph{cp} est une commande permettant de copier des fichier.
Cette commande prend en paramètres deux nom de fichier, le premier est le fichier \emph{source}, fichier qui doit être copié. Le second est la destination : il s'agit soit un dossier, auquel cas le nom de fichier sera préservé, soit d'un chemin vers un fichier (qui sera écrasé si il existe déjà).\\
Afin de copier tout un dossier, le mode récursif de \emph{cp} est utilisé, pour cela l'on ajoute le modificateur de commande "-r" qui permettra la copie de tous les répertoires, sous répertoires et fichiers contenu depuis la source jusqu'à la destination.
Sans ce modificateur, la commande cp omettra la copie de dossier.

\subsubsection*{Exemples d'utilisations}

\begin{lstlisting}[caption=copie de fichier]
~/$ ls
miaou.txt chat.c poulpe.exe
~/$ cp miaou.txt chaton.txt
~/$ ls
miaou.txt chaton.txt chat.c poulpe.exe
~/$
\end{lstlisting}

\begin{lstlisting}[language=bash,caption=copie de dossier]
~/$ ls
Chaton/
~/$ ls Chaton/
miaou.txt chat.c poulpe.exe
~/$ cp -r Chaton/ Miaou/
~/$ ls
Chaton/ Miaou/
~/$ ls Miaou/
miaou.txt chat.c poulpe.exe
~/$
\end{lstlisting}
