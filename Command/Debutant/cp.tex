\subsection{cp}

\subsubsection*{Description}

\paragraph{} \texttt{cp} est une commande permettant de copier des fichiers.
Cette commande prend en paramètres deux noms de fichier. Le premier est le
fichier \emph{source}, fichier qui doit être copié. Le second est la
destination : il s'agit soit un dossier, auquel cas le nom de fichier sera
préservé, soit d'un chemin vers un fichier (qui sera écrasé si il existe
déjà).

\paragraph{} Afin de copier tout un dossier, il faut copier les fichiers et les
sous-dossiers qui sont dedans, c'est à dire copier de manière \emph{récursive}.
On utilise alors l'option \texttt{-r} de \texttt{cp}. Sans ce modificateur, la
commande cp omettra la copie de dossier.

\subsubsection*{Exemples d'utilisation}

\begin{lstlisting}[caption=copie de fichier]
~/$ ls
miaou.txt chat.c poulpe.exe
~/$ cp miaou.txt chaton.txt
~/$ ls
miaou.txt chaton.txt chat.c poulpe.exe
~/$
\end{lstlisting}

\begin{lstlisting}[language=bash,caption=copie de dossier]
~/$ ls
Chaton/
~/$ ls Chaton/
miaou.txt chat.c poulpe.exe
~/$ cp -r Chaton/ Miaou/
~/$ ls
Chaton/ Miaou/
~/$ ls Miaou/
miaou.txt chat.c poulpe.exe
~/$
\end{lstlisting}
