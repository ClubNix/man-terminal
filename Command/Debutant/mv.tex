\subsection{mv}

\subsubsection*{Description}

\paragraph{}
\texttt{mv} sert à deplacer des fichiers, il peut aussi tout un dossier sans
besoin de rajouter d'options. Lors du déplacement, on peut aussi donner un
nouveau nom au fichier, ainsi pour renommer un fichier, on le déplace au même
endroit. Son utilisation est similaire à la commande \texttt{cp} (hormis le
\texttt{-r} qui n'est pas nécéssaire).

\subsubsection*{Exemple d'utilisation}

\begin{lstlisting}[caption=déplacement d'un fichier]
$ ls
miaou.txt chat.c poulpe.exe Chaton/
$ ls Chaton/

$ mv miaou.txt Chaton/
$ ls Chaton/
miaou.txt
$ ls
chat.c poulpe.exe
\end{lstlisting}

\begin{lstlisting}[caption=renommer un fichier]
$ ls
miaou.txt chat.c poulpe.exe Chaton/
$ mv miaou.txt chat.txt
$ ls
chat.txt chat.c poulpe.exe Chaton/
\end{lstlisting}
