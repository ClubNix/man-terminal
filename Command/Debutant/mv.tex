\subsection{mv}

\subsubsection*{Description}

\paragraph{} \texttt{mv} sert à déplacer des fichiers, il peut aussi déplacer
tout un dossier sans avoir besoin de rajouter d'options. Lors du déplacement,
on peut aussi donner un nouveau nom au fichier. Ainsi pour renommer un fichier,
on le déplace au même endroit. Son utilisation est similaire à la commande
\texttt{cp} (hormis le \texttt{-r} qui n'est pas nécéssaire).

\subsubsection*{Exemple d'utilisation}

\begin{lstlisting}[caption=déplacement d'un fichier]
$ ls
texte.txt code.c image.png Sous-Dossier/
$ ls Sous-Dossier/

$ mv texte.txt Sous-Dossier/
$ ls Chaton/
texte.txt
$ ls
code.c image.png Sous-Dossier/
\end{lstlisting}

\begin{lstlisting}[caption=renommer un fichier]
$ ls
texte.txt code.c image.png Sous-Dossier/
$ mv texte.txt citation.txt
$ ls
citation.txt code.c image.png Sous-Dossier/
\end{lstlisting}
