\subsection{cd}
\subsubsection{Description}
\emph{cd} permet de naviguer a travers l'arborescence des dossiers.
En plus de pouvoir se déplacer dans un fichier, cette commande accepte plusieurs caractères spéciaux.
En essayant de se déplacer dans le dossier "..", la commande permet de revenir au dossier précédent.
Il aussi est possible de revenir directement à la racine de ses fichiers, sa "home" à l'aide du caractère "\~", à noter qu'omettre un nom de dossier a la même conséquence.

La commande cd accepte des chemins \emph{absolus}, "/home/nepta/Poly\textbackslash Linux/", mais aussi des chemins \emph{relatifs}, "Poly\textbackslash \textvisiblespace Linux" permettant de se déplacer en fonction du dossier courant.

Certains caractères, notamment le caractère espace (" ") doivent être \emph{échapper} afin d'être pris en compte, pour cela le caratère "\textbackslash" est utilisé ("Poly Linux" devient "Polyt\textbackslash Linux",
"Chaton\textbackslash Chat\textbackslash Tigre" devient "Chaton\textbackslash \textbackslash Chat\textbackslash \textbackslash Tigre")

\subsubsection{Exemples d'utilisations}

\begin{lstlisting}
~/$ cd Poly\ Linux
~/Poly Linux/$ cd Chaton/Petit\ Chat/
~/Poly Linux/Chaton/Petit Chat/$ cd
~/ $ 
\end{lstlisting}
