\subsection*{./exec}
\subsubsection*{Description}
Afin d'executer un programme, l'interpreteur de commande doit comprendre que le programme se situe dans un dossier et non dans la liste des commandes installé.
Pour ce faire on peut lui donner un le chemin absolue vers le logiciel ou relatif, taper directement \emph{exec} sera cependant considéré comme une commande à chercher dans une destination standard (là où se trouve les autres commandes cat,ls ...).
On utilise donc le fichier special \emph{.}, representant le dossier courrant, pour finalement obtenir \emph{./exec} signifiant donc \emph{"le fichier exec se situant dans le dossier courrant"}.

\subsubsection*{Exemple d'utilisation}
\begin{lstlisting}
$ ./poulpe.exe
oh, un çhat!
$ 
\end{lstlisting}
