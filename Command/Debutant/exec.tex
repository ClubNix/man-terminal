\subsection{./exec}

\subsubsection*{Description}

\paragraph{} Afin d'exécuter un programme, le shell doit comprendre que le
programme se situe dans un dossier et non dans la liste des commandes
installées.  Pour ce faire, on peut lui donner le chemin absolu ou relatif vers
le logiciel. Taper directement \emph{exec} sera cependant considéré comme une
commande à chercher dans une destination standard (là où se trouvent les autres
commandes \texttt{cat}, \texttt{ls}\ldots).  On utilise donc le fichier spécial
\emph{.}, représentant le dossier courant, pour finalement obtenir
\emph{./exec}, signifiant donc ``\emph{l'exécutable exec se situant dans le
	dossier courant}''.

\subsubsection*{Exemple d'utilisation}
\begin{lstlisting}
$ ./monExecutable
Hello, World!
$ 
\end{lstlisting}
