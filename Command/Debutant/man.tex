\subsection{man}

\subsubsection*{Description}

\paragraph{} \emph{man} sert à consulter le manuel du système. Il s'agit là
probablement de la commande la plus utile pour un débutant dans un terminal.
Elle prend en paramètre une ou plusieurs commandes dont on veut connaître la
description et l'utilisation.

\subsubsection*{Exemple d'utilisation}

\begin{lstlisting}
$ man cal
CAL(1)                   User Commands                   CAL(1)



NAME
       cal - display a calendar

SYNOPSIS
       cal [options] [[[day] month] year]

DESCRIPTION
       cal  displays  a  simple  calendar.  If no arguments are
       specified, the current month is displayed.

OPTIONS
       -1, --one
              Display  single  month  output.   (This  is   the
              default.)

       -3, --three
              Display three months spanning the date.

       -n , --months number
              Display number of months, starting from the month
              containing the date.
...
\end{lstlisting}

\paragraph{} Parmi les informations importantes ici, on peut voir le
\emph{synopsis} de la commande. Les mots entre crochets sont facultatifs. Cela
signifie donc que toutes les options (commençant par des tirets) sont
facultatives, et que spécifier l'année est facultatif, de même que le mois si
l'année est présente, etc\ldots

\paragraph{} Par exemple, on peut écrire comme commandes:
\begin{itemize}
	\item \texttt{cal}, ce qui donne:
\begin{lstlisting}
   septembre 2015
lu ma me je ve sa di
    1  2  3  4  5  6
 7  8  9 10 11 12 13
14 15 16 17 18 19 20
21 22 23 24 25 26 27
28 29 30
\end{lstlisting}
	\item \texttt{cal -1}, qui nous donne le même résultat que la commande
		précédente
	\item \texttt{cal 2015}, qui nous donne le calendrier pour 2015
		(un peu long).
	\item \texttt{cal ----months 4 02 2014} (affiche 4 mois à partir de février
		2014), nous donnant donc:
\begin{lstlisting}[basicstyle=\footnotesize\ttfamily]
    février 2014            mars 2014            avril 2014
lu ma me je ve sa di  lu ma me je ve sa di  lu ma me je ve sa di
                1  2                  1  2      1  2  3  4  5  6
 3  4  5  6  7  8  9   3  4  5  6  7  8  9   7  8  9 10 11 12 13
10 11 12 13 14 15 16  10 11 12 13 14 15 16  14 15 16 17 18 19 20
17 18 19 20 21 22 23  17 18 19 20 21 22 23  21 22 23 24 25 26 27
24 25 26 27 28        24 25 26 27 28 29 30  28 29 30
                      31
      mai 2014
lu ma me je ve sa di
          1  2  3  4
 5  6  7  8  9 10 11
12 13 14 15 16 17 18
19 20 21 22 23 24 25
26 27 28 29 30 31
\end{lstlisting}
\end{itemize}
