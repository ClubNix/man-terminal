\subsection{man}

\subsubsection*{Description}

\paragraph{} \emph{man} sert à consulter le manuel du système. Il s'agit là
probablement de la commande la plus utile pour un débutant dans un terminal.
Elle prend en paramètre une ou plusieurs commandes dont on veut connaître la
description et l'utilisation.

\subsubsection*{Exemple d'utilisation}

\begin{lstlisting}
$ man cat
CAL(1)                   User Commands                   CAL(1)



NAME
       cal - display a calendar

SYNOPSIS
       cal [options] [[[day] month] year]

DESCRIPTION
       cal  displays  a  simple  calendar.  If no arguments are
       specified, the current month is displayed.

OPTIONS
       -1, --one
              Display  single  month  output.   (This  is   the
              default.)

       -3, --three
              Display three months spanning the date.

       -n , --months number
              Display number of months, starting from the month
              containing the date.
...
\end{lstlisting}

\paragraph{} Parmi les informations importantes ici, on peut voir le synopsis
de la commande. Les mots entre crochets sont facultatifs. Cela signifie donc
que toutes les options (commençant par des tirets) sont facultatifs, et que
spécifier l'année est facultative, de même pour le mois si l'année est
présente, etc\ldots

\paragraph{} Par exemple, on peut écrire comme commandes:
\begin{itemize}
	\item \texttt{cal}
	\item \texttt{cal -1}
	\item \texttt{cal 2015}
	\item \texttt{cal ----months 4 02 2014} (affiche 4 mois à partir de février
		2014)
\end{itemize}
