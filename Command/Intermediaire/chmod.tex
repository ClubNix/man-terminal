\subsection*{chmod/chown}
\subsubsection*{Description}
\noindent \emph{chmod} pour \emph{ch}ange file \emph{mod}e bits et\\
\emph{chown} pour  \emph{ch}ange file \emph{own}er and group\\
permettent de changer les attributs d'un fichier, c'est à dire qui peut lire écrire ou exécuter le fichier.
A noter que seul l'utilisateur root peut modifier le propriétaire (owner) d'un fichier.
Pour les attributs, ils se séparent en trois catégories \textbf{u}ser, \textbf{g}roup et \textbf{o}thers.
l'user correspond aux droits du propriétaire du fichier, group aux utilisateur appartenant au même groupe que ce fichier et others, tout le monde.\\
Pour chacune de ces catégorie 3 attributs existent: \textbf{r}ead (droit en lecture), \textbf{w}rite (droit en écriture) et e\textbf{x}ecute (droit d’exécution ou de traverser si c'est un dossier)

\noindent L'utilisation de chown est assez simple:
\begin{lstlisting}
chown <nouveau propriétaire>:<nouveau groupe> fichier
chown <nouveau propriétaire> fichier
chown :<nouveau groupe> fichier
\end{lstlisting}

\noindent
Pour chmod, il faut savoir compter en base 8. Après avoir fait un "ls -l" on peut apercevoir une ligne ressemblant à ça:
\begin{lstlisting}
-rw-r--r-- 1 club nix 13 sept.  2 05:15 chaton
\end{lstlisting}
la partie à regarder est "rw-r--r--" qui peut être (si tous les droits sont accordé à tout le monde) "rwxrwxrwx".
Pour obtenir la représentation en octal (base 8), il faut convertir chaque triplet "rwx" en somme de chiffres avec la règle suivante:
r vaut 4, w vaut 2 et x vaut 1, on a ainsi par exemple, pour les droits "rw-", 4+2 = 6
ainsi les droits lecture/écriture pour l'utilisateur et lecture pour le groupe et les autres correspond à "644"

\begin{lstlisting}
$ ls -l 777.txt
-rwxrwxrwx 1 club nix 0 sept.  2 05:44 777.txt
$ chmod 644 777.txt
$ ls -l 777.txt
-rw-r--r-- 1 club nix 0 sept.  2 05:44 777.txt
\end{lstlisting}

\subsubsection*{Exemple d'utilisation}
\begin{lstlisting}
$ ls -l
total 16
-rwxrwxrwx 1 club nix    0 sept.  2 05:44 777.txt
-rw-r--r-- 1 club nix 13 sept.  2 05:15 chaton
-r--r--r-- 1 club nix  0 sept.  2 05:14 miaou
-rw-r--r-- 1 club nix 8518 sept.  2 05:27 poulpe.exe
$ cat chaton
oh! un çhat
$ chmod -r miaou
$ cat chaton
cat: chaton: Permission denied
$ echo "miaou" >> miaou 
-bash: miaou: Permission denied
$ ./poulpe.exe 
-bash: ./poulpe.exe: Permission denied
$ chmod u+x poulpe.exe
$ ./poulpe.exe
oh, un çhat
\end{lstlisting}
