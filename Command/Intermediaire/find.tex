\subsection*{find}
\subsubsection*{Description}

\paragraph{}
\emph{find} est un utilitaire qui, comme son nom l'indique, permet de trouver
certains fichiers ayant des caractéristiques particulières. Il permet par
exemple de trouver des fichiers ayant un nom particulier, ou une taille plus
grande que 5Mo, créés avant/après une certaine date ou une combinaison de tout
cela et plus encore.

\paragraph{}
La syntaxe basique de la commande \emph{find} est de type:
\begin{lstlisting}
find chemin [options]
\end{lstlisting}

\paragraph{}
Pour les options numériques, il est possible de spécifier \textit{n} pour une
valeur égale à $n$, \textit{+n} pour une valeur supérieure à $n$ et
\textit{-n} pour une valeur inférieure à $n$.\\

Parmi les options, on trouve par exemple :

\begin{table}[h]
	\centering
    \begin{tabular}{|l|p{10cm}|}
        \hline
        \textbf{Option}              & \textbf{Description}\\
        \hline
        -name \textit{"nom"}         & Le fichier a un nom de la forme
                                       \textit{"nom"} (les patterns avec * et
									   autres sont autorisés).\\
        \hline
        -iname \textit{"nom"}        & Équivalent à -name mais insensible à la
		                               casse.\\
        \hline
        -path \textit{"chemin"}      & Le chemin du fichier est de la forme
                                       \textit{"chemin"}.\\
        \hline
        -size \textit{n}             & Le fichier est de taille $n$
                                       (\textit{c} pour octet, \textit{k} pour
									   kilo octets, \textit{M} pour mega
									   octets, \textit{G} pour giga octets).\\
        \hline
        -ctime \textit{n}            & Le fichier a été changé pour la dernière
		                               fois il y a $n$ jours. \\
        \hline
        -user \textit{"utilisateur"} & Le fichier appartient à
                                       \textit{"utilisateur"}.\\
		\hline
    \end{tabular}
	\caption{Exemples d'options de la commande \emph{find}}
	\label{tab:find-opts}
\end{table}

\paragraph{}
Il est aussi possible de spécifier à \emph{find} des actions à exécuter pour
chaque fichier trouvé, comme par exemple les supprimer ou les afficher d'une
manière particulière.

Parmi les actions, on trouve :

\begin{table}[h]
	\centering
    \begin{tabular}{|l|p{10cm}|}
        \hline
        \textbf{Action} & \textbf{Description}\\
        \hline
        -delete                   & Supprime les fichiers trouvés. Attention
                                    -delete ne demande aucune confirmation et
									n'affiche pas les fichiers supprimés.\\
        \hline
        -printf \textit{"format"} & Affiche le texte de forme
                                    \textit{"format"} avec les escapes codes
									(\texttt{\textbackslash n} et autres) et en
									remplaçant par exemple \texttt{\%f} par le
									nom de fichier ou dossier et \texttt{\%h}
									par le chemin menant au fichier ou
									dossier.\\
		\hline
		-print0                   & Affiche le nom du fichier avec le chemin en
                                    en entier et le caractère null
									(\texttt{\textbackslash 0}) à la fin de
									chaque. Utile quand utilisé avec la
									commande \texttt{xargs} (voir exemples).\\
        \hline
		-exec \textit{commande} ; & Exécute la commande \textit{commande} pour
                                    chaque fichier en remplaçant \texttt{\{\}}
									par le nom avec chemin du fichier en
									cours.\\
		\hline
		-ok \textit{commande} ;   & Équivalent à -exec mais demande la
                                    permission l'utilisateur avant.\\
		\hline
    \end{tabular}
	\caption{Exemples d'actions de la commande \emph{find}}
	\label{tab:find-acts}
\end{table}

\pagebreak
\paragraph{}
Pour une liste plus étendue des options et des actions de \emph{find}, utilisez
la commande \lstinline|man 1 find|.

\subsubsection*{Exemple d'utilisation}

\begin{lstlisting}
$ find . -name "*.conf" # Cherche tout les fichiers dont le nom se termine par .conf dans le dossier courant et ses sous-dossiers
./.tmux.conf
./.fonts.conf.d/10-powerline-symbols.conf
$ find . -size +5M # Cherche tout les fichiers dont la taille est supérieure à 5 mega octets
./Images/wallpaper.png
./Vidéos/Test.mkv
./Vidéos/Usavitch.mp4
$ find / -name "*windows*" -delete -print # Supprime tout les fichiers ayant "windows" dans leur nom et les affiche depuis la racine
/tmp/hate-windows.pdf
$ find ~ -size +10M -ok rm {} \textbackslash; # Cherche tout les fichiers de taille supérieure à 10 mega octet et demande à l'utilisateur s'il veut le supprimer
< file ... ~/Vidéos/Usavitch.mp4 > ? n
$ find ~/Développement/C/Projet1/ -name "*.c" -print0 | xargs -0 grep main # Cherche la chaine de caractère "main" dans tout les fichiers du dossier ~/Développement/C/Projet1 qui se terminent par .c
main.c:int main (int argc, char const* argv[]) {
\end{lstlisting}
