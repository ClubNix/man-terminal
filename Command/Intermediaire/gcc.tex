\subsection*{gcc}
\subsubsection*{Description}
\emph{gcc} pour \textbf{G}NU \textbf{C} \textbf{C}ompiler est sans doute le logiciel que vous utiliserez le plus souvent sur votre système.
C'est le compilateur distribué par défaut avec les distributions GNU/Linux, au centre de beaucoup de projets son utilisation est complexe et possède plus de 12000 lignes de manuel.
Nous ne vous présenterons donc ici que les options les plus utiles,
pour plus d'informations, renseignez vous sur les makefile.

\subsubsection*{Exemple d'utilisation}

\begin{lstlisting}
$ ls
chat.c
$ gcc chat.c #compiler un fichier (sortie a.out)
$ gcc chat.c -o poulpe.exe #compiler un fichier (sortie poulpe.exe)
$ gcc *.c -o poulpe.exe #compiler tous les fichier .c du répertoire
$ gcc *.c -O2 #activer les optimisation 
$ gcc *.c -g #activer les symbole de débug
$ gcc *.c -Wall #activer tous les warning (vous prévient si vous faites des bêtises)
$ gcc *.c -std=c99 #choisi le standard C à utiliser (par défaut gnu90)
\end{lstlisting}

pour les optimisations il existe aussi -O1 plus rapide mais moins optimisé, -O0 sans optimisation (utile pour le débugge)  -Os les plus petites sources possible et -Ofast le plus rapidement possible (la vitesse de compilation pas d'exécution)

Bien sûr les différent argument peuvent être combinés dans l'ordre que l'on veut pour donner une commande qui compile tous les fichier C du répertoire courant avec le maximum d'information, les symboles de debug et une version plus souple du C (en c99 pas besoin de déclarer toutes les variable en début de block et possibilité d'utiliser la syntaxe java pour les boucle for)
\begin{lstlisting}
$ gcc -Wall -Wextra -O0 -g -std=c99 *.c -o exec
\end{lstlisting}
