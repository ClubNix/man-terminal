\subsection*{grep}
\subsubsection*{Description}

\paragraph{}
\emph{grep}, pour \emph{\textbf{g}lobally search a \textbf{r}egular
\textbf{e}xpression and \textbf{p}rint} est une des commandes les plus utiles
qui soit.  Elle permet en effet d'effectuer des recherche dans un document
selon un certain \emph{pattern}, c'est-à-dire la recherche d'une expression
quelconque (mot se finissant par -ou, commençant par mia- contenant\ldots).

En plus de pouvoir faire ces recherches dans les documents, grep peut aussi en
faire sur l'entrée standard, et son utilité en devient accrue en la chaînant
aux autres commandes à l'aide d'une pipe (``|''), permettant de filtrer une
sortie pour mettre en évidence les données utiles.

Il existe une multitude d'options pour cette commande, nous ne retiendrons que
les principales:
\begin{itemize}
\item[-n] permet d'afficher la ligne et le fichier dans lequel l'expression à
	été trouvée
\item[-i] désactive la sensibilité à la casse (les lettres majuscules et
	minuscules sont traitées indifféremment)
\item[-A|B|C] permet de conserver plus d'une ligne avant/après l'expression
	\begin{itemize}
		\item[-A\emph{X}] garde X lignes \textbf{après} l'expression
		\item[-B\emph{X}] garde X lignes \textbf{avant} l'expression
		\item[-C\emph{X}] garde X lignes \textbf{avant} et \textbf{après}
			l'expression
	\end{itemize}
\item[-e] permet de rajouter des pattern à la recherche, -e \emph{pattern1} -e
	\emph{pattern2} cherchera à la fois le pattern1 et le pattern2 dans le
	fichier
\item[-{}-color] permet de colorer les patterns trouvés
\end{itemize}

\subsubsection*{Exemple d'utilisation}

\begin{lstlisting}
$ echo -e "miaou\nchaton\nchat" |grep miaou
miaou
$ ls
notes.txt
$ cat notes.txt |grep -n cirno
     9  cirno 9/9
$ grep -n -e trax -e nepta notes.txt
2:nepta 9/20
6:trax over 9000!
\end{lstlisting}
