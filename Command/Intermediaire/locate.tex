\subsection*{locate}
\subsubsection*{Description}
\emph{locate} sert a localiser (chercher) des fichier. Après avoir utiliser sa commande sœur \emph{updatedb} permettant de générer l'index de recherche, il suffit de lui donner en argument un nom de fichier a rechercher.
A noter que locate est sensible à la casse (a moins d'utiliser l'option \emph{-i}) et recherche tous les fichier contenant l'argument (pour la recherche de \emph{chat}, \emph{chat}on sera trouvé).
Locate recherchant dans un index généré par updatedb aucun fichier crée après le ré-indexation du système (après avoir exécuter updatedb) ne sera trouvé, l'emplacement par défaut de l'index se trouvant dans un emplacement privilégié (appartient à root), seul root peut lancer cette commande (updatedb)

\subsubsection*{Exemple d'utilisation}

\begin{lstlisting}
$ sudo updatedb
$ locate locate
(...)
/usr/bin/locate
(...)
$
\end{lstlisting}
