\subsection*{df/du}
\subsubsection*{Description}

\paragraph{}
\emph{df} et \emph{du} sont deux commandes permettant d'obtenir des
informations sur votre système de fichier.  \emph{df} affiche de manière
globale l'utilisation du disque (\% d'espace libre) alors que \emph{du} permet
d'obtenir l'espace occupé par un dossier.

\noindent Pour ces deux commandes l'utilisation de l'argument "-h" est
recommandée (affichage lisible par un humain).

\subsubsection*{Exemple d'utilisation}

\begin{lstlisting}
$ df -h
Sys. de fichiers Taille Utilisé Dispo Uti% Monté sur
/dev/sda3          145G    3,8G  134G   3% /
tmpfs              198M    356K  198M   1% /run
udev                10M       0   10M   0% /dev
shm                987M    4,0K  987M   1% /dev/shm
cgroup_root         10M       0   10M   0% /sys/fs/cgroup
\end{lstlisting}

Ici seule la première ligne est importante, les autres systèmes de fichiers
étant "spéciaux" (générés par l'OS et montés dans la RAM).  Elle nous indique
la taille de la partition, la quantité d'espace utilisé (c'est pas beaucoup
parce que je suis sur un serveur venant d'être installé), l'espace disque total
et un ratio des deux.  La colonne "monté sur" correspond à l'emplacement de
votre système de fichier, vous pouvez obtenir par exemple
"/media/club/USB-stick" pour des clés usb.

\begin{lstlisting}
$ cd /boot/
$ du -h *
1,6M    System.map-3.16.1-gentoo
71K     config-3.16.1-gentoo
710K    grub/locale
1,9M    grub/i386-pc
5,1M    grub
12K     lost+found
3,3M    vmlinuz-3.16.1-gentoo
\end{lstlisting}
