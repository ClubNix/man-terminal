\subsection*{kill}
\subsubsection*{Description}

\paragraph{}

\emph{kill} et \emph{killall} sont des commandes qui servent à \textit{tuer} un
processus ou de manière plus simple arrêter un programme. Les options de
\emph{kill} et \emph{killall} permettent d'envoyer des ``signaux'' au programme
qui permettent de le tuer de manière plus ou moins ``gentille''. Ainsi, envoyer
le signal de ``fin'' (SIGTERM) est le même que celui qui est envoyé lorsque
l'on clique sur la croix d'un programme alors que le signal de ``mort''
(SIGKILL) permet de terminer le programme sans avertissement ni sauvegarde de
documents non enregistrés.

\paragraph{}
Voici un tableau des signaux les plus utilisés et leur signification
(une description plus complète est disponible via la commande
\lstinline|man 7 signal|) :

\begin{table}[h]
	\centering
	\begin{tabular}{|l|c|l|}
		\hline
		\textbf{Nom du signal} & \textbf{Valeur} & \textbf{Signification} \\
		\hline
		SIGHUP & 1 & Redémarre le programme sans préavis \\
		\hline
		SIGKILL & 9 & Tue le programme (sans préavis) \\
		\hline
		SIGTERM & 15 & Ferme le programme normalement (par défaut dans
		\emph{kill} et \emph{killall}) \\
		\hline
	\end{tabular}
	\caption{Principaux signaux envoyables via \emph{kill} ou \emph{killall}}
	\label{tab:signal}
\end{table}

\subsubsection*{Exemple d'utilisation}

\begin{lstlisting}
$ kill 12345 # Termine le programme ayant pour PID (Process Id) 1234
$ killall -9 gedit # Tue tous les processus ayant pour nom "gedit"
\end{lstlisting}
