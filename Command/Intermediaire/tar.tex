\subsection*{tar}
\subsubsection*{Description}

\paragraph{}

\emph{tar} est un utilitaire qui permet de créer et de gérer des archives, soit
de compresser et / ou regrouper des fichiers dans un autre fichier, de les
lire, ou encore de les extraire. La syntaxe est de la sorte :

\begin{lstlisting}
tar action [fichiers/options]
\end{lstlisting}

Parmis les actions, on trouve :

\begin{table}[h]
	\centering
	\begin{tabular}{|c|p{10cm}|}
		\hline
		\textbf{Action} & \textbf{Description}\\
		\hline
		c               & Crée une archive\\
		\hline
		x               & Extrait une archive\\
		\hline
		t               & Liste les fichiers présents dans une archive\\
		\hline
	\end{tabular}
	\caption{Liste des actions possibles pour la commande \emph{tar}}
	\label{tab:taractions}
\end{table}

Cependant, il faut pour spécifier une archive, ou dire sous quelle format de
compression l'on veut effectuer l'action, utiliser d'autres options. Parmis
elles, on trouve:

\begin{table}[h]
	\centering
	\begin{tabular}{|c|p{10cm}|}
		\hline
		\textbf{Action} & \textbf{Description}\\
		\hline
		f & Utilise l'archive suivante (pour création, modification, listage,
			etc\ldots)\\
		\hline
		v & Mode verbose, affiche les fichiers en train d'être traités.\\
		\hline
		a & Auto détecte la compression utilisée.\\
		\hline
		z & Utilise la compression \emph{gzip}.\\
		\hline
	\end{tabular}
	\caption{Liste des options possibles pour la commande \emph{tar}}
	\label{tab:taroptions}
\end{table}

Ainsi donc, pour créer une archive s'appelant ``\texttt{archive.tar}'',
contenant les fichiers ``\texttt{1.txt}'' et ``\texttt{2.txt}'', il suffit
d'utiliser la commande :

\begin{lstlisting}
tar cf archive.tar 1.txt 2.txt
\end{lstlisting}

ou 
\begin{lstlisting}
tar cvf archive.tar 1.txt 2.txt
\end{lstlisting}

pour afficher en plus les fichiers en train d'être traités

\subsubsection*{Exemple d'utilisation}

\begin{lstlisting}
$ tar cf archive.tar 1.txt 2.txt
$ tar cvf archive.tar 1.txt 2.txt
1.txt
2.txt
$ tar cvzf archive.tar.gz *.txt
1.txt
2.txt
3.txt
hello.txt
$ tar caf archive.tar.xz *.txt
$ tar tvf archive.tar.xz
-rw-rw-r-- user/club 349379 2014-10-22 13:47 1.txt
-rw-rw-r-- user/club 349390 2014-10-22 13:50 2.txt
-rw-rw-r-- user/club 349410 2014-10-22 13:59 3.txt
-rw-rw-r-- user/club 349123 2014-10-21 14:33 hello.txt
$ tar xvaf archive.tar.xz
1.txt
2.txt
3.txt
hello.txt
\end{lstlisting}
