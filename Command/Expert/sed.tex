\subsection*{sed}
\subsubsection*{Description}
\emph{sed}, non pas pour \emph{search and destroy} mais \emph{\textbf{s}tream \textbf{ed}itor} est un utilitaire permettant -- entre autre -- de substituer du texte.
Dans son usage le plus courant, il est l'équivalent de la fonction \emph{rechercher et remplacer} de la plupart des éditeurs de texte.
Cependant couplé aux regexp, il devient un outils beaucoup plus puissant que la plupart de ces fonctions, de plus en étant utilisé dans le terminal, il peut effectuer ses substitution sur plusieurs fichier et automatiser la substitution de texte sur tout un projet.
Son utilisation dans la forme la plus simple est la suivante:
\begin{quote}
sed -i s/\emph{pattern}/\emph{substitution string}/g
\end{quote}

Le \emph{pattern} est du même type que ceux rencontré avec la commande grep, et pour chaque occurrence sera remplacer par la \emph{substitution string}, enfin la modificateur \emph{g} permet de rendre la substitution \textbf{g}lobal (par défaut seul la première occurrence est changé).
Enfin, le texte résultant est afficher sur la sortie standard, et le modificateur \emph{-i} doit être appliqué pour effectuer les modification sur place (c-à-d directement dans le fichier lu)

\subsubsection*{Exemple d'utilisation}

\begin{lstlisting}
$ cat chat.txt
les chats sont des animaux à poils
les chats sont mignons
les chats ont 4 pattes
$ sed chat.txt s/chat/chaton/g
les chatons sont des animaux à poils
les chatons sont mignons
les chatons ont 4 pattes
\end{lstlisting}

Ici, le "s" n'est pas remplacé et est gardé entre le passage de chat à chaton.
