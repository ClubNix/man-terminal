\subsection*{git}
\subsubsection*{Description}
\emph{git} est un un logiciel de gestion de versions distribué, c'est-à-dire un logiciel qui permet d'enregistrer l'évolution d'un ou plusieurs fichiers au cours  du temps. Le terme \emph{distribué} signifie que chaque utilisateur possède sa propre copie du \emph{dépôt} (projet), ce qui permet entre autres de ne pas dépendre d'un serveur contenant un dépôt principal comme c'est le cas avec un système \emph{centralisé}.\\

Un dépôt peut être créé (\lstinline{git init}) ou cloné depuis un projet déjà existant (\lstinline{git clone nomUtilisateur@hote:/chemin/vers/répertoire}). A chaque fois que vous effectuez des modifications sur les fichiers de votre projet, celles-ci doivent être enregistrées. Ci-dessous le cycle de vie d'un ou plusieurs fichier(s) :
\begin{description}
\item[1. non-suivi] un fichier non-suivi n'est pas "versionné" par git;
\item[2. suivi] le fichier a été ajouté à l'index de git par l'utilisateur (\lstinline{git add});
\item[3. non modifié] le fichier n'a pas encore été modifié;
\item[4. modifié] le fichier a été modifié par l'utilisateur;
\item[5. indexé \& modifié] le fichier a été modifié et ajouté à l'index (\lstinline{git add});
\item[6. enregistré] les modifications ont été sauvegardées (\lstinline{git commit}).
\end{description}
\textbf{N.B.} Certains fichiers peuvent être ignorés à l'aide du fichier \emph{.gitgnore}.\\

\noindent Quelques bonnes pratiques peuvent être suivies au niveau des commits:
\begin{itemize}
\item un commit devrait contenir des modifications liées. Par exemple, la résolution de deux bugs devrait se faire dans deux commits différents;
\item Un commit ne devrait pas contenir un travail à moitié effectué. Si besoin est, des modifications peuvent être mises de côté à l'aide de \lstinline{git stash};
\item Chaque commit doit avoir un message qui fasse un bref résumé des modifications (il faut éviter les messages du type: fix, fix bug, etc).
\end{itemize}
\hfill{}

Une des fonctionnalités de Git est la gestion de branches. Un dépôt Git possède de base une branche principale appelée \emph{master}. L'utilisateur peut ajouter ses propres branches, ce qui lui permet de travailler sur plusieurs aspects de son projet (ex: correction d'un bug, ajout d'une fonctionnalité) tout en gardant une branche principale fonctionnelle. Une fois les modifications effectuées sur une branche créée par l'utilisateur, celle-ci peut être fusionnée dans la branche principale (merge).

\subsubsection*{Exemples d'utilisation}
\paragraph{Premier exemple (dépôt)}
\begin{enumerate}
\item Création d'un dépôt git
\begin{lstlisting}
$ git init test
Initialized empty Git repository in /home/user/test/.git/
$ cd test
\end{lstlisting}
\item Ajout d'un fichier
\begin{lstlisting}
$ echo -e '#!bin/sh\necho miaou' > test.sh
$ git status
On branch master

Initial commit

Untracked files:
  (use "git add <file>..." to include in what will be committed)

	test.sh

nothing added to commit but untracked files present (use "git add" to track)
\end{lstlisting}
\item Ajoute le fichier à l'index de git
\begin{lstlisting}
$ git add test.sh
$ git status
On branch master

Initial commit

Changes to be committed:
  (use "git rm --cached <file>..." to unstage)

	new file:   test.sh
\end{lstlisting}
\item Valider les modifications
\begin{lstlisting}
$ git commit -m 'first file in the project!'
[master (root-commit) 0f162a5] first file in the project!
 1 file changed, 2 insertions(+)
 create mode 100644 test.sh
$ git status
On branch master
nothing to commit, working directory clean
\end{lstlisting}
\end{enumerate}

\paragraph{Second exemple (dépôt distant)}
\begin{enumerate}
\item Clone le dépôt git man-terminal hébergé sur GitHub
\begin{lstlisting}
$ git clone https://github.com/ClubNix/man-terminal
\end{lstlisting}
\item modifications de fichiers, indexation et commit;
\begin{lstlisting}
...
$ git add 
$ git commit -m 'test'
\end{lstlisting}
\item envoi des modifications sur le serveur distant (ici GitHub)
\begin{lstlisting}
$ git push
\end{lstlisting}

\item récupère les modifications que d'autres contributeurs ont ajoutés sur le dépôt distant
\begin{lstlisting}
...
$ git pull
\end{lstlisting}
\end{enumerate}

\paragraph{Troisième exemple (branches \& stash)}
\begin{enumerate}
\item un développeur travaille sur un programme;
\item il souhaite ajouter une fonctionnalité "faire le café" à celui-ci, il créé ainsi une branche pour travailler sur cette fonctionnalité;
\begin{lstlisting}
$ git checkout -b faireCafe
Switched to a new branch 'faireCafe'
\end{lstlisting}
\item découverte d'un bug majeur, le développeur souhaite interrompre pour le moment le développement de la fonctionnalité "faire le café". Copie les modifications non terminées dans un répertoire de travail à part (\lstinline{git stash});
\begin{lstlisting}
...
$ git stash
Saved working directory and index state WIP on master: c6d69c2 test
HEAD is now at c6d69c2 test
\end{lstlisting}
\item Retour sur la branche principale;
\begin{lstlisting}
$ git checkout master
Switched to branch 'master'
\end{lstlisting}
\item création d'une nouvelle branche pour travailler sur un correctif;
\begin{lstlisting}
$ git checkout -b paymentFix
Switched to a new branch 'paymentFix'
\end{lstlisting}
\item développement du correctif terminé, fusion de la branche du correctif dans la branche principale;
\begin{lstlisting}
...
$ git checkout master
$ git merge paymentFix
\end{lstlisting}
\item retour sur la branche de la fonctionnalité "faire le café" et restaure les modifications sauvegardées dans le stash.
\begin{lstlisting}
$ git checkout faireCafe
$ git stash apply
\end{lstlisting}
\end{enumerate}

\subsubsection*{Liens utiles}
\begin{itemize}
\item \href{http://rogerdudler.github.io/git-guide/}{git - the simple guide}
\item \href{http://git-scm.com/book/fr}{git Book}
\item \href{http://stackoverflow.com/questions/315911/git-for-beginners-the-definitive-practical-guide#320140}{Git for beginners: The definitive practical guide}
\end{itemize}