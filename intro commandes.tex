\documentclass[french, a4paper, 12pt, titlepage]{article}
%% Peut remplacer "article" par "scrartcl" %%

\usepackage{a4wide}
%\usepackage[top=2cm, bottom=2cm, left=2cm, right=2cm]{geometry}
\raggedbottom % prevents vertical white space on pages that cannot be filled properly

\usepackage{hyperref}
\hypersetup{
	colorlinks=true,       	% false: boxed links; true: colored links
	linkcolor=black,          	% color of internal links
	urlcolor=blue,           	% color of external links
	citecolor=grey
}

\usepackage[T1]{fontenc}
%\usepackage{fourier}
%\usepackage{utopia}
%\usepackage{palatino}

\usepackage{lmodern}
%% ajouter fonte petite capitale grasse à lmodern avec celle de computer modern %%
\rmfamily
\DeclareFontShape{T1}{lmr}{b}{sc}{<->ssub*cmr/bx/sc}{}
\DeclareFontShape{T1}{lmr}{bx}{sc}{<->ssub*cmr/bx/sc}{}
%% /ajout %%
\usepackage{wrapfig}

%\usepackage[a4paper]{geometry} % marges plus petites que a4paper standard
\usepackage{listings} % insérer code source
\usepackage{algorithm} % algorithmique
\usepackage{algorithmic}
\usepackage{url}
\usepackage[usenames, dvipsnames]{color} % couleurs (nombre de base étendu)
\usepackage{graphicx} % insérer images
\usepackage[utf8]{inputenc}
\usepackage[french]{babel}
\usepackage{amsmath}
\usepackage{amsfonts}
\usepackage{amssymb}
\usepackage{amsthm}
\usepackage{multicol}
\definecolor{grey}{rgb}{0.96,0.96,0.96}
\definecolor{grey2}{rgb}{0.3,0.3,0.3}

%% Define listings params %%
\lstset{
	numbers=left,
	language=Java,
	tabsize=4,
	frame=single, % cadre autour du code
	breaklines=true, % autorise couper ligne trop longue
	basicstyle=\small\ttfamily,
	numberstyle=\scriptsize\ttfamily,
	backgroundcolor=\color{grey},
	showstringspaces=false,
	keywordstyle=\color{OliveGreen},
	stringstyle=\color{BrickRed},
	commentstyle=\color{grey2}\it,
	stepnumber=1
} % numérote toute les x lignes
% listing utf8 fr %
\lstset{%
	inputencoding=utf8,
	extendedchars=true,
	literate=
		{é}{{\'{e}}}1
		{è}{{\`{e}}}1
		{ê}{{\^{e}}}1
		{ë}{{\¨{e}}}1
		{û}{{\^{u}}}1
		{ù}{{\`{u}}}1
		{â}{{\^{a}}}1
		{à}{{\`{a}}}1
		{î}{{\^{i}}}1
		{ç}{{\c{c}}}1
		{Ç}{{\c{C}}}1
		{É}{{\'{E}}}1
		{Ê}{{\^{E}}}1
		{À}{{\`{A}}}1
		{Â}{{\^{A}}}1
		{Î}{{\^{I}}}1
}
%% /Define listings params %%

%% Francisation des algorithmes
\renewcommand{\algorithmicrequire} {\textbf{\textsc{Entrées:}}}
\renewcommand{\algorithmicensure}  {\textbf{\textsc{Sorties:}}}
\renewcommand{\algorithmicwhile}   {\textbf{tant que}}
\renewcommand{\algorithmicdo}      {\textbf{faire}}
\renewcommand{\algorithmicendwhile}{\textbf{fin tant que}}
\renewcommand{\algorithmicend}     {\textbf{fin}}
\renewcommand{\algorithmicif}      {\textbf{si}}
\renewcommand{\algorithmicendif}   {\textbf{fin si}}
\renewcommand{\algorithmicelse}    {\textbf{sinon}}
\renewcommand{\algorithmicthen}    {\textbf{alors}}
\renewcommand{\algorithmicfor}     {\textbf{pour}}
\renewcommand{\algorithmicforall}  {\textbf{pour tout}}
\renewcommand{\algorithmicdo}      {\textbf{faire}}
\renewcommand{\algorithmicendfor}  {\textbf{fin pour}}
\renewcommand{\algorithmicloop}    {\textbf{boucler}}
\renewcommand{\algorithmicendloop} {\textbf{fin boucle}}
\renewcommand{\algorithmicrepeat}  {\textbf{répéter}}
\renewcommand{\algorithmicuntil}   {\textbf{jusqu'à}}
\renewcommand{\algorithmiccomment} {\STATE //}
\newcommand{\BEGIN}{\STATE \fbox{Début}}
\newcommand{\END}{\STATE \fbox{Fin}}
\floatname{algorithm}{Algorithme}
%% /francisation des algorithmes

\renewcommand{\qedsymbol}{}

\newcommand{\petit}[1]{
	\medskip \noindent
	\begin{small}
	#1)
	\end{small}
}

\begin{document}

\title{Introduction aux commandes shell (Linux) basique}
\author{Club*Nix}
\date{Compilé le \today}

\maketitle
%% Laisse page blanche pour verso page de garde %%

\vfill
\pagebreak

%\tableofcontents
\newpage
\strut\thispagestyle{empty}
\vfill
\pagebreak
\tableofcontents
\strut\thispagestyle{empty}
%\setcounter{page}{0}
\newpage
\setcounter{page}{1}

\section{Presentation d'un environnement linux}
TODO: historique linux
-> annecdote linus et pinguin

\section{Qu'est-ce que le shell}
(photo coquillage)

\section{Commandes Basique (débutant)}
\subsection{Liste de commandes}
\begin{description}
\item[cd] \emph{\textbf{c}hange \textbf{d}irectory} permet de naviguer à travers les répertoires
\item[ls] \emph{\textbf{l}i\textbf{s}t directory contents} permet d'afficher le contenu d'un répertoire
\item[cat] \emph{con\textbf{cat}enate files} affiche un fichier sur l'entrée standard
\item[./<exec>] lance le script/programme \emph{<exec>}
\item[cp] \emph{\textbf{c}o\textbf{p}y files} copie un fichier
\item[mv] \emph{\textbf{m}o\textbf{v}e file} déplace un fichier
\item[man] \emph{reference \textbf{man}uals} affiche un manuel sur une commande, fonction, bibliothèque (essayez "man man")
\end{description}

\section{Liste de commande étendue (intermédiaire)}
\begin{enumerate}
\item[make/gcc]
\item[du]
\item[df]
\item[touch]
\item[find]
\item[locate]
\item[chmod]
\item[chown]
\item[kill]
\item[nano]
\item[tar]
\item[wget]
\item[flux redirection]
\end{enumerate}
regexp simple, logiciel sympa (tmux, mocp, lua/bash/python),

\section{Regexp (expert)}
\begin{enumerate}
\item[mount]
\item[ncdu]
\item[halt/reboot]
\item[find]
\item[ps]
\item[awk]
\item[sed]
\item[ssh]
\item[git]
\item[usermod]
\item[groupmod]
\item[ifconfig]
\item[grep]
\end{enumerate}
un peu de réseaux? services,
dossier de base (/etc/, /var/, /home/, /mnt/)

\section{Une interface graphique c'est bien aussi}
lolwoot, y'a vraiment des gens qui font des apt-get install? 

\end{document}
