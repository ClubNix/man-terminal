\documentclass[french, a4paper, 12pt, titlepage]{article}
%% Peut remplacer "article" par "scrartcl" %%

\usepackage{a4wide}
%\usepackage[top=2cm, bottom=2cm, left=2cm, right=2cm]{geometry}
\raggedbottom % prevents vertical white space on pages that cannot be filled properly

\usepackage{hyperref}
\hypersetup{
	colorlinks=true,       	% false: boxed links; true: colored links
	linkcolor=black,          	% color of internal links
	urlcolor=blue,           	% color of external links
	citecolor=grey
}

\usepackage[T1]{fontenc}
%\usepackage{fourier}
%\usepackage{utopia}
%\usepackage{palatino}

\usepackage{lmodern}
%% ajouter fonte petite capitale grasse à lmodern avec celle de computer modern %%
\rmfamily
\DeclareFontShape{T1}{lmr}{b}{sc}{<->ssub*cmr/bx/sc}{}
\DeclareFontShape{T1}{lmr}{bx}{sc}{<->ssub*cmr/bx/sc}{}
%% /ajout %%
\usepackage{wrapfig}

%\usepackage[a4paper]{geometry} % marges plus petites que a4paper standard
\usepackage{listings} % insérer code source
\usepackage{algorithm} % algorithmique
\usepackage{algorithmic}
\usepackage{url}
\usepackage[usenames, dvipsnames]{color} % couleurs (nombre de base étendu)
\usepackage{graphicx} % insérer images
\usepackage[utf8]{inputenc}
\usepackage[french]{babel}
\usepackage{amsmath}
\usepackage{amsfonts}
\usepackage{amssymb}
\usepackage{amsthm}
\usepackage{multicol}
\definecolor{grey}{rgb}{0.96,0.96,0.96}
\definecolor{grey2}{rgb}{0.3,0.3,0.3}

%% Define listings params %%
\lstset{
	numbers=left,
	language=Java,
	tabsize=4,
	frame=single, % cadre autour du code
	breaklines=true, % autorise couper ligne trop longue
	basicstyle=\small\ttfamily,
	numberstyle=\scriptsize\ttfamily,
	backgroundcolor=\color{grey},
	showstringspaces=false,
	keywordstyle=\color{OliveGreen},
	stringstyle=\color{BrickRed},
	commentstyle=\color{grey2}\it,
	stepnumber=1
} % numérote toute les x lignes
% listing utf8 fr %
\lstset{%
	inputencoding=utf8,
	extendedchars=true,
	literate=
		{é}{{\'{e}}}1
		{è}{{\`{e}}}1
		{ê}{{\^{e}}}1
		{ë}{{\¨{e}}}1
		{û}{{\^{u}}}1
		{ù}{{\`{u}}}1
		{â}{{\^{a}}}1
		{à}{{\`{a}}}1
		{î}{{\^{i}}}1
		{ç}{{\c{c}}}1
		{Ç}{{\c{C}}}1
		{É}{{\'{E}}}1
		{Ê}{{\^{E}}}1
		{À}{{\`{A}}}1
		{Â}{{\^{A}}}1
		{Î}{{\^{I}}}1
}
%% /Define listings params %%

%% Francisation des algorithmes
\renewcommand{\algorithmicrequire} {\textbf{\textsc{Entrées:}}}
\renewcommand{\algorithmicensure}  {\textbf{\textsc{Sorties:}}}
\renewcommand{\algorithmicwhile}   {\textbf{tant que}}
\renewcommand{\algorithmicdo}      {\textbf{faire}}
\renewcommand{\algorithmicendwhile}{\textbf{fin tant que}}
\renewcommand{\algorithmicend}     {\textbf{fin}}
\renewcommand{\algorithmicif}      {\textbf{si}}
\renewcommand{\algorithmicendif}   {\textbf{fin si}}
\renewcommand{\algorithmicelse}    {\textbf{sinon}}
\renewcommand{\algorithmicthen}    {\textbf{alors}}
\renewcommand{\algorithmicfor}     {\textbf{pour}}
\renewcommand{\algorithmicforall}  {\textbf{pour tout}}
\renewcommand{\algorithmicdo}      {\textbf{faire}}
\renewcommand{\algorithmicendfor}  {\textbf{fin pour}}
\renewcommand{\algorithmicloop}    {\textbf{boucler}}
\renewcommand{\algorithmicendloop} {\textbf{fin boucle}}
\renewcommand{\algorithmicrepeat}  {\textbf{répéter}}
\renewcommand{\algorithmicuntil}   {\textbf{jusqu'à}}
\renewcommand{\algorithmiccomment} {\STATE //}
\newcommand{\BEGIN}{\STATE \fbox{Début}}
\newcommand{\END}{\STATE \fbox{Fin}}
\floatname{algorithm}{Algorithme}
%% /francisation des algorithmes

\renewcommand{\qedsymbol}{}

\newcommand{\petit}[1]{
	\medskip \noindent
	\begin{small}
	#1)
	\end{small}
}

\begin{document}

\title{Introduction aux commandes shell (Linux) basique}
\author{Club*Nix}
\date{Compilé le \today}

\maketitle
%% Laisse page blanche pour verso page de garde %%

\vfill
\pagebreak

%\tableofcontents
\newpage
\strut\thispagestyle{empty}
\vfill
\pagebreak
\tableofcontents
\strut\thispagestyle{empty}
%\setcounter{page}{0}
\newpage
\setcounter{page}{1}

\section{Presentation d'un environnement linux}
TODO: historique linux
-> annecdote linus et pinguin

\section{Qu'est-ce que le shell}
(photo coquillage)
Le Shell, ou coquillage correspond à l'interpréteur de commande utilisé à l'intérieur de votre terminal.
Il en existe de toute sorte (bash, tcsh, sh, ksh, zsh ...) tous permettent d'exécuter des commandes, programmes <insert text here>.
Dans le monde réel, \emph{bash} est l'interpréteur le plus courant, cependant pour quelques raison obscure vous commencerais par default votre aventure à l'ESIEE avec \emph{tcsh}.
Très peu convivial, nous supposerons pour la suite que vous utilisez bash (vous pouvez lancer le shell \emph{bash} à l'intérieur de \emph{tcsh} en tapant la commande "bash" à l'affichage de votre \emph{prompt}).
Votre terminal est donc composé d'une fenêtre, permettant d'afficher un historique de commande, de menu pour configurer les couleurs, la font ... et enfin du \emph{shell}, l'interpréteur qui est lancé automatiquement.
Ce que vous pouvez voir a partir de la est une ligne ressemblant à ceci:
\begin{lstlisting}
club@nix ~/$ _
\end{lstlisting}
\begin{tiny}
Je ne peut pas faire de clignotement sur ce poly, je vous prie d'imaginer l'underscore clignoter
\end{tiny}\\
Vous avez donc un affichage d'un certain nombre d'information a travers votre \emph{prompt}.
\begin{enumerate}
\item[club@nix] correspond a votre login (nom d'utilisateur, ici "club") suivit du nom de la machine (ici "nix") sur laquelle vous vous trouvez, le tout est séparé du caractère "@".
\item[$\sim$/] S'en suis ensuite votre dossier courant suivit du caractère "\$".
\item[\$\_] Enfin la ligne de commande, vous permettant de taper des ... commandes.
\end{enumerate}

\section{Notation}
\begin{enumerate}
\item[\^{}] correspond à la touche "\emph{Ctrl}", \^{}C, signfie donc qu'il faut appuyer en même temps sur la touche "Ctrl" et C (deux touches sont appuyé).
\end{enumerate}

\section{Commandes Basiques (débutant)}
Voici votre première expérience du terminal!
Il peut vous faire peur et picoter les yeux mais n'ayez crainte. Tout d'abord commençons par ce picotement des yeux, en faisant un clique droit et allant dans profils, préférence du profil vous pourrez changer la couleurs pour quelque chose de moins agressif pour vos yeux, "blanc sur noir" ou "vert sur noir" (pour les plus geeks d'entre vous) plusieurs choses peuvent être configuré, je vous laisse les découvrir.


\subsection{Liste de commandes}
\begin{description}
\item[cd] \emph{\textbf{c}hange \textbf{d}irectory} permet de naviguer à travers les répertoires
\item[ls] \emph{\textbf{l}i\textbf{s}t directory contents} permet d'afficher le contenu d'un répertoire
\item[cat] \emph{con\textbf{cat}enate files} affiche un fichier sur l'entrée standard
\item[./<exec>] lance le script/programme \emph{<exec>}
\item[cp] \emph{\textbf{c}o\textbf{p}y files} copie un fichier
\item[mv] \emph{\textbf{m}o\textbf{v}e file} déplace un fichier
\item[man] \emph{reference \textbf{man}uals} affiche un manuel sur une commande, fonction, bibliothèque (essayez "man man")
\end{description}

\subsection{Cat}
\subsubsection{Description}
\emph{Cat} est une commande permettant d'afficher du texte sur l'entrée standard.
Elle prend en entrée un fichier: texte, code source, script... pour l'afficher directement sur le terminal.
Attention toutefois, en l'utilisant sur un fichier binaire, la commande s'exécutera et en essayant d'afficher le contenu du fichier souvent trop gros pour votre terminal, et un fichier binaire correspondant a une suite aléatoire de caractère ascii, cassera l'affichage de celui ci.
Si par inadvertance cela vous arrivait, vous pouvez toujours essayer de taper "reset" pour rétablir votre shell (pas d'inquiétude si vous ne voyez rien, la commande est quand même écrite correctement).

\subsubsection{Exemple d'utilisation}

\begin{lstlisting}
$ls
chaton.txt chat.c poulpe.exe
$cat chat.c
#include <stdio.h>

int main(){
	printf("oh, un çhat!");
 return 0;
}
$
\end{lstlisting}

\subsection*{cd}
\subsubsection*{Description}
\emph{cd} permet de naviguer à travers l'arborescence des dossiers.
En plus de pouvoir se déplacer dans un fichier, cette commande accepte plusieurs caractères spéciaux.
En essayant de se déplacer dans le dossier "..", la commande permet de revenir au dossier précédent.
Il est aussi possible de revenir directement à la racine de ces fichiers, sa "home" à l'aide du caractère "\~", à noter qu'omettre un nom de dossier a la même conséquence.

La commande cd accepte des chemins \emph{absolus}, "/home/nepta/Poly\textbackslash Linux/", mais aussi des chemins \emph{relatifs}, "Poly\textbackslash \textvisiblespace Linux" permettant de se déplacer en fonction du dossier courant.

Certains caractères, notamment le caractère espace (" ") doivent être \emph{échappés} afin d'être pris en compte, pour cela le caractère "\textbackslash" est utilisé ("Poly Linux" devient "Polyt\textbackslash Linux",
"Chaton\textbackslash Chat\textbackslash Tigre" devient "Chaton\textbackslash \textbackslash Chat\textbackslash \textbackslash Tigre")

\subsubsection*{Exemples d'utilisations}

\begin{lstlisting}
~/$ cd Poly\ Linux
~/Poly Linux/$ cd Chaton/Petit\ Chat/
~/Poly Linux/Chaton/Petit Chat/$ cd
~/ $ 
\end{lstlisting}

\subsection{cp}
\subsubsection{Description}
\emph{cp} est une commande permettant de copier des fichier.
Cette commande prend en paramètres deux nom de fichier, le premier est le fichier \emph{source}, fichier qui doit être copié, le second, la destination qui est soit un dossier auquel cas le nom de fichier sera préservé, soit un chemin vers un fichier (qui sera écraser si il existe déjà).\\
Afin de copier tout un dossier, le mode récursif de \emph{cp} est utilisé, pour cela l'on ajoute le modificateur de commande "-r" qui permettra la copie de tous les répertoire, sous répertoire et fichier contenu depuis la source jusqu'à la destination.
Sans se modificateur, la commande cp ometra la copie de dossier.

\subsubsection{Exemples d'utilisations}

\begin{lstlisting}[caption=copie de fichier]
~/$ ls
miaou.txt chat.c poulpe.exe
~/$ cp miaou.txt chaton.txt
~/$ ls
miaou.txt chaton.txt chat.c poulpe.exe
~/$
\end{lstlisting}

\begin{lstlisting}[language=bash,caption=copie de dossier]
~/$ ls
Chaton/
~/$ ls Chaton/
miaou.txt chat.c poulpe.exe
~/$ cp -r Chaton/ Miaou/
~/$ ls
Chaton/ Miaou/
~/$ ls Miaou/
miaou.txt chat.c poulpe.exe
~/$
\end{lstlisting}

\subsection*{ls}
\subsubsection*{Description}
\emph{ls} liste les fichiers et dossiers présents dans le répertoire courant.
Il existe plusieurs options utiles à cette commande comme \emph{-l} qui donne des informations détaillées sur les fichiers.
Une autre option utile est \emph{-{}-color} qui permet d'obtenir des indications sur les fichiers en colorant les noms de ceux-ci, les fichiers exécutables deviennent vert, les dossiers bleus ...
Pour ceux qui se souviennent encore de \emph{DOS}, cette commande est l'equivalent (quoique beaucoup plus complète) de \emph{dir}, il en existe d'ailleurs un alias sur les ordinateurs à l'esiee \textcolor{grey2}{à vérifier}

\subsubsection*{Exemple d'utilisation}
\begin{lstlisting}
$ ls
Chaton/
$ ls Chaton/
miaou.txt chat.c poulpe.exe
$ ls -l Chaton/
total 20
-rw-r--r-- 1 club user    13 déc.   6 01:32 miaou.txt
-rwxr-xr-x 1 club user 10019 déc.   6 01:33 pouple.exe
-rw-r--r-- 1 club user   301 déc.   6 01:33 value.c
\end{lstlisting}

\subsection{./exec}

\subsubsection*{Description}

\paragraph{}
Afin d'executer un programme, l'interpréteur de commande doit comprendre que le
programme se situe dans un dossier et non dans la liste des commandes
installées.  Pour ce faire on peut lui donner le chemin absolu vers le logiciel
ou relatif, taper directement \emph{exec} sera cependant considéré comme une
commande à chercher dans une destination standard (là où se trouve les autres
commandes cat,ls ...).  On utilise donc le fichier special \emph{.},
représentant le dossier courant, pour finalement obtenir \emph{./exec}
signifiant donc \emph{"l'exécutable exec se situant dans le dossier courant"}.

\subsubsection*{Exemple d'utilisation}
\begin{lstlisting}
$ ./poulpe.exe
oh, un çhat!
$ 
\end{lstlisting}

\subsection{mv}

\subsubsection*{Description}

\paragraph{} \texttt{mv} sert à déplacer des fichiers, il peut aussi déplacer
tout un dossier sans avoir besoin de rajouter d'options. Lors du déplacement,
on peut aussi donner un nouveau nom au fichier. Ainsi pour renommer un fichier,
on le déplace au même endroit. Son utilisation est similaire à la commande
\texttt{cp} (hormis le \texttt{-r} qui n'est pas nécéssaire).

\subsubsection*{Exemple d'utilisation}

\begin{lstlisting}[caption=déplacement d'un fichier]
$ ls
texte.txt code.c image.png Sous-Dossier/
$ ls Sous-Dossier/

$ mv texte.txt Sous-Dossier/
$ ls Chaton/
texte.txt
$ ls
code.c image.png Sous-Dossier/
\end{lstlisting}

\begin{lstlisting}[caption=renommer un fichier]
$ ls
texte.txt code.c image.png Sous-Dossier/
$ mv texte.txt citation.txt
$ ls
citation.txt code.c image.png Sous-Dossier/
\end{lstlisting}

\subsection*{man}
\subsubsection*{Description}
\emph{man} sert à consulter le manuel du système. 

\subsubsection*{Exemple d'utilisation}

\begin{lstlisting}
$ man man
$ man cat
$ man rsync
\end{lstlisting}


\section{Liste de commande étendue (intermédiaire)}
\begin{enumerate}
\item[make/gcc]
\item[du]
\item[df]
\item[touch]
\item[find]
\item[locate]
\item[chmod]
\item[chown]
\item[kill]
\item[nano]
\item[tar]
\item[wget]
\item[flux redirection]
\end{enumerate}
regexp simple, logiciel sympa (tmux, mocp, lua/bash/python),

\section{Regexp (expert)}
\begin{enumerate}
\item[mount]
\item[ncdu]
\item[halt/reboot]
\item[find]
\item[ps]
\item[awk]
\item[sed]
\item[ssh]
\item[git]
\item[usermod]
\item[groupmod]
\item[ifconfig]
\item[grep]
\end{enumerate}
un peu de réseaux? services,
dossier de base (/etc/, /var/, /home/, /mnt/)

\section{Une interface graphique c'est bien aussi}
lolwoot, y'a vraiment des gens qui font des apt-get install? 

\end{document}
