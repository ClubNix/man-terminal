\documentclass[french, a4paper, 12pt, titlepage]{article}
%% Peut remplacer "article" par "scrartcl" %%

\usepackage{a4wide}
%\usepackage[top=2cm, bottom=2cm, left=2cm, right=2cm]{geometry}
\raggedbottom % prevents vertical white space on pages that cannot be filled properly

\usepackage{hyperref}
\hypersetup{
	colorlinks=true,       	% false: boxed links; true: colored links
	linkcolor=black,          	% color of internal links
	urlcolor=blue,           	% color of external links
	citecolor=grey
}

\usepackage[T1]{fontenc}
%\usepackage{fourier}
%\usepackage{utopia}
%\usepackage{palatino}

\usepackage{lmodern}
%% ajouter fonte petite capitale grasse à lmodern avec celle de computer modern %%
\rmfamily
\DeclareFontShape{T1}{lmr}{b}{sc}{<->ssub*cmr/bx/sc}{}
\DeclareFontShape{T1}{lmr}{bx}{sc}{<->ssub*cmr/bx/sc}{}
%% /ajout %%
\usepackage{wrapfig}

%\usepackage[a4paper]{geometry} % marges plus petites que a4paper standard
\usepackage{listings} % insérer code source
\usepackage{algorithm} % algorithmique
\usepackage{algorithmic}
\usepackage{url}
\usepackage[usenames, dvipsnames]{color} % couleurs (nombre de base étendu)
\usepackage{graphicx} % insérer images
\usepackage[utf8]{inputenc}
\usepackage[french]{babel}
\usepackage{amsmath}
\usepackage{amsfonts}
\usepackage{amssymb}
\usepackage{amsthm}
\usepackage{multicol}
\definecolor{grey}{rgb}{0.96,0.96,0.96}
\definecolor{grey2}{rgb}{0.3,0.3,0.3}

%% Define listings params %%
\lstset{
	numbers=left,
	language=Java,
	tabsize=4,
	frame=single, % cadre autour du code
	breaklines=true, % autorise couper ligne trop longue
	basicstyle=\small\ttfamily,
	numberstyle=\scriptsize\ttfamily,
	backgroundcolor=\color{grey},
	showstringspaces=false,
	keywordstyle=\color{OliveGreen},
	stringstyle=\color{BrickRed},
	commentstyle=\color{grey2}\it,
	stepnumber=1
} % numérote toute les x lignes
% listing utf8 fr %
\lstset{%
	inputencoding=utf8,
	extendedchars=true,
	literate=
		{é}{{\'{e}}}1
		{è}{{\`{e}}}1
		{ê}{{\^{e}}}1
		{ë}{{\¨{e}}}1
		{û}{{\^{u}}}1
		{ù}{{\`{u}}}1
		{â}{{\^{a}}}1
		{à}{{\`{a}}}1
		{î}{{\^{i}}}1
		{ç}{{\c{c}}}1
		{Ç}{{\c{C}}}1
		{É}{{\'{E}}}1
		{Ê}{{\^{E}}}1
		{À}{{\`{A}}}1
		{Â}{{\^{A}}}1
		{Î}{{\^{I}}}1
}
%% /Define listings params %%

%% Francisation des algorithmes
\renewcommand{\algorithmicrequire} {\textbf{\textsc{Entrées:}}}
\renewcommand{\algorithmicensure}  {\textbf{\textsc{Sorties:}}}
\renewcommand{\algorithmicwhile}   {\textbf{tant que}}
\renewcommand{\algorithmicdo}      {\textbf{faire}}
\renewcommand{\algorithmicendwhile}{\textbf{fin tant que}}
\renewcommand{\algorithmicend}     {\textbf{fin}}
\renewcommand{\algorithmicif}      {\textbf{si}}
\renewcommand{\algorithmicendif}   {\textbf{fin si}}
\renewcommand{\algorithmicelse}    {\textbf{sinon}}
\renewcommand{\algorithmicthen}    {\textbf{alors}}
\renewcommand{\algorithmicfor}     {\textbf{pour}}
\renewcommand{\algorithmicforall}  {\textbf{pour tout}}
\renewcommand{\algorithmicdo}      {\textbf{faire}}
\renewcommand{\algorithmicendfor}  {\textbf{fin pour}}
\renewcommand{\algorithmicloop}    {\textbf{boucler}}
\renewcommand{\algorithmicendloop} {\textbf{fin boucle}}
\renewcommand{\algorithmicrepeat}  {\textbf{répéter}}
\renewcommand{\algorithmicuntil}   {\textbf{jusqu'à}}
\renewcommand{\algorithmiccomment} {\STATE //}
\newcommand{\BEGIN}{\STATE \fbox{Début}}
\newcommand{\END}{\STATE \fbox{Fin}}
\floatname{algorithm}{Algorithme}
%% /francisation des algorithmes

\renewcommand{\qedsymbol}{}

\newcommand{\petit}[1]{
	\medskip \noindent
	\begin{small}
	#1)
	\end{small}
}

\begin{document}

\title{Introduction aux commandes shell (Linux) basique}
\author{Club*Nix}
\date{Compilé le \today}

\maketitle
%% Laisse page blanche pour verso page de garde %%

\vfill
\pagebreak

%\tableofcontents
\newpage
\strut\thispagestyle{empty}
\vfill
\pagebreak
\tableofcontents
\strut\thispagestyle{empty}
%\setcounter{page}{0}
\newpage
\setcounter{page}{1}

\section{Presentation d'un environnement linux}
TODO: historique linux
-> annecdote linus et pinguin

\section{Qu'est-ce que le shell}
(photo coquillage)

\section{Commandes Basique (débutant)}
\subsection{Liste de commandes}
\begin{description}
\item[cd] \emph{\textbf{c}hange \textbf{d}irectory} permet de naviguer à travers les répertoires
\item[ls] \emph{\textbf{l}i\textbf{s}t directory contents} permet d'afficher le contenu d'un répertoire
\item[cat] \emph{con\textbf{cat}enate files} affiche un fichier sur l'entrée standard
\item[./<exec>] lance le script/programme \emph{<exec>}
\item[cp] \emph{\textbf{c}o\textbf{p}y files} copie un fichier
\item[mv] \emph{\textbf{m}o\textbf{v}e file} déplace un fichier
\item[man] \emph{reference \textbf{man}uals} affiche un manuel sur une commande, fonction, bibliothèque (essayez "man man")
\end{description}

\subsection{Cat}
\subsubsection{Description}
\emph{Cat} est une commande permettant d'afficher du texte sur l'entrée standard.
Elle prend en entrée un fichier: texte, code source, script... pour l'afficher directement sur le terminal.

Attention toutefois : lors de son utilisation sur un fichier binaire, la commande s'exécutera bien. Cependant, le contenu du fichier est souvent trop gros pour votre terminal, et la suite de caracteres ASCII aléatoires à laquelle il correspond cassera votre terminal lorsque cat essaiera de l'afficher.
Si par inadvertance cela vous arrivait, vous pouvez toujours essayer de taper "reset" pour rétablir votre shell (il n'y a pas d'inquiétude à avoir si vous ne voyez pas la commande "reset" s'afficher, les caractères tapés sont bien pris en compte).


\subsubsection{Exemple d'utilisation}

\begin{lstlisting}
$ls
chaton.txt chat.c poulpe.exe
$cat chat.c
#include <stdio.h>

int main(){
	printf("oh, un chat!");
 return 0;
}
$
\end{lstlisting}

\subsection{cd}
\subsubsection*{Description}

\paragraph{} \texttt{cd} permet de naviguer à travers l'arborescence des
dossiers. Comme paramètre, cette commande accepte aussi bien un chemin relatif
qu'absolu. Sans paramètre, cette commande vous mènera à votre dossier personnel
(\texttt{\~}).

\subsubsection*{Exemples d'utilisation}

\begin{lstlisting}
~/$ cd Documents
~/Documents/$ cd ESIEE/man-terminal
~/Documents/ESIEE/man-terminal/$ cd
~/$
\end{lstlisting}

\subsection*{cp}
\subsubsection*{Description}
\emph{cp} est une commande permettant de copier des fichiers.
Cette commande prend en paramètres deux nom de fichier, le premier est le fichier \emph{source}, fichier qui doit être copié. Le second est la destination : il s'agit soit un dossier, auquel cas le nom de fichier sera préservé, soit d'un chemin vers un fichier (qui sera écrasé si il existe déjà).\\
Afin de copier tout un dossier, le mode récursif de \emph{cp} est utilisé, pour cela on ajoute le modificateur de commande "-r" qui permettra la copie de tous les répertoires, sous-répertoires et fichiers contenus depuis la source jusqu'à la destination.
Sans ce modificateur, la commande cp omettra la copie de dossier.

\subsubsection*{Exemples d'utilisation}

\begin{lstlisting}[caption=copie de fichier]
~/$ ls
miaou.txt chat.c poulpe.exe
~/$ cp miaou.txt chaton.txt
~/$ ls
miaou.txt chaton.txt chat.c poulpe.exe
~/$
\end{lstlisting}

\begin{lstlisting}[language=bash,caption=copie de dossier]
~/$ ls
Chaton/
~/$ ls Chaton/
miaou.txt chat.c poulpe.exe
~/$ cp -r Chaton/ Miaou/
~/$ ls
Chaton/ Miaou/
~/$ ls Miaou/
miaou.txt chat.c poulpe.exe
~/$
\end{lstlisting}

\subsection{ls}
\subsubsection*{Description}

\paragraph{} \texttt{ls} liste les fichiers et dossiers présents dans le
répertoire courant. Il prend en paramètre un dossier dont on veut lister les
fichiers. Sans paramètre cette commande liste les fichiers contenus dans le
répertoire courant. Il existe plusieurs options utiles à cette commande, comme
\texttt{-l} qui donne des informations détaillées sur les fichiers. Une autre
option utile est \texttt{----color} qui permet d'obtenir des indications sur
les fichiers en colorant leurs noms; les fichiers exécutables
deviennent vert, les dossiers bleus\ldots

\subsubsection*{Exemple d'utilisation}
\begin{lstlisting}
$ ls
Dossier/
$ ls Dossier/
texte.txt code.c image.png
$ ls -l Dossier/
total 20
-rw-r--r-- 1 club user    13 déc.   6 01:32 texte.txt
-rwxr-xr-x 1 club user 10019 déc.   6 01:33 image.png
-rw-r--r-- 1 club user   301 déc.   6 01:33 code.c
\end{lstlisting}

\paragraph{} On voit que l'option \texttt{-l} affiche plein d'informations
utiles. La première colonne indique les droits sur le fichier, que l'on
expliquera après, la troisième colonne avec \texttt{club} indique le
propriétaire du fichier (donc le propriétaire est l'utilisateur \texttt{club}).
La quatrième colonne représente le groupe propriétaire du fichier. Dans cet
exemple, le groupe propriétaire est le groupe \texttt{user}, qui est le groupe
des utilisateurs normaux. La cinquième colonne est la taille du fichier en
octets. Attention, la taille affichée par \texttt{ls -l} pour les dossiers ne
correspond pas à la taille de tous les sous-fichiers, et sous-dossiers. La
sixième colonne est la date de dernière modification du fichier (à savoir ici
le 6 décembre 1h33).

\paragraph{} Pour la première colonne, il faut savoir que les fichiers sous
GNU/Linux ont trois permissions principales: le droit de lecture (représenté
par \texttt{r} pour \emph{read}), le droit d'écriture (représenté par
\texttt{w} pour \emph{write}) et le droit d'exécution (représenté par
\texttt{x} pour \emph{eXecute}).

\paragraph{} La première lettre de cette colonne correspond au type de fichier.
Si par exemple, il s'agit d'un dossier, on verra la lettre \texttt{d}.  Et si
c'est un fichier normal, ce sera un \texttt{-}. Les trois prochaines lettres
correspondent respectivement au droit de lecture, écriture et exécution pour le
propriétaire du fichier: si la lettre (\texttt{r}, \texttt{w} ou \texttt{x})
est présente, cela veut dire que le propriétaire possède le droit, sinon, la
lettre sera remplacée par un \texttt{-}. Les trois lettres suivantes sont pour
les membres du groupe auquel appartient le fichier, et les trois dernières
lettres sont pour les autres utilisateurs.

\paragraph{} Par exemple, si l'on a \texttt{-rw-r----r----}, cela veut dire
qu'il s'agit d'un fichier normal et que le propriétaire peut lire le fichier,
le modifier (et donc le supprimer) mais pas l'exécuter. Les membres du même
groupe et les autres utilisateurs peuvent seulement le lire. Autre exemple: si
l'on a \texttt{drwxr-xr-x}, on a là un dossier qui peut être lu (donc voir son
contenu), écrit (rajouter des fichiers dedans) et exécuté (pouvoir ``rentrer''
dans le dossier) par le propriétaire. Les membres du groupe et les autres
utilisateurs peuvent seulement le lire et l'exécuter.

\subsection{./exec}

\subsubsection*{Description}

\paragraph{} Afin d'exécuter un programme, le shell doit comprendre que le
programme se situe dans un dossier et non dans la liste des commandes
installées.  Pour ce faire, on peut lui donner le chemin absolu ou relatif vers
le logiciel, taper directement \emph{exec} sera cependant considéré comme une
commande à chercher dans une destination standard (là où se trouvent les autres
commandes cat, ls\ldots).  On utilise donc le fichier spécial \emph{.},
représentant le dossier courant, pour finalement obtenir \emph{./exec}
signifiant donc \emph{"l'exécutable exec se situant dans le dossier courant"}.

\subsubsection*{Exemple d'utilisation}
\begin{lstlisting}
$ ./poulpe.exe
oh, un çhat!
$ 
\end{lstlisting}

\subsection*{mv}
\subsubsection*{Description}
\emph{mv} sert à deplacer des fichiers, il peut aussi à l'aide de l'option \emph{-r} déplacer tout un dossier.
Lors du déplacement, on peut aussi donner un nouveau nom au fichier, ainsi pour renommer un fichier, on le déplace au même endroit.
Son utilisation est similaire à la commande \emph{cd}

\subsubsection*{Exemple d'utilisation}

\begin{lstlisting}[caption=déplacement d'un fichier]
$ ls
miaou.txt chat.c poulpe.exe Chaton/
$ ls Chaton/

$ mv miaou.txt Chaton/
$ ls Chaton/
miaou.txt
$ ls
chat.c poulpe.exe
\end{lstlisting}

\begin{lstlisting}[caption=renommer un fichier]
$ ls
miaou.txt chat.c poulpe.exe Chaton/
$ mv miaou.txt chat.txt
$ ls
chat.txt chat.c poulpe.exe Chaton/
\end{lstlisting}

\subsection{man}

\subsubsection*{Description}

\paragraph{} \texttt{man} sert à consulter le manuel du système. Il s'agit là
probablement de la commande la plus utile pour un débutant dans un terminal.
Elle prend en paramètre une ou plusieurs commandes dont on veut connaître la
description et l'utilisation.

\subsubsection*{Exemple d'utilisation}

\begin{lstlisting}
$ man cal
CAL(1)                   User Commands                   CAL(1)



NAME
       cal - display a calendar

SYNOPSIS
       cal [options] [[[day] month] year]

DESCRIPTION
       cal  displays  a  simple  calendar.  If no arguments are
       specified, the current month is displayed.

OPTIONS
       -1, --one
              Display  single  month  output.   (This  is   the
              default.)

       -3, --three
              Display three months spanning the date.

       -n , --months number
              Display number of months, starting from the month
              containing the date.
...
\end{lstlisting}

\paragraph{} Parmi les informations importantes, on peut voir le
\emph{synopsis} de la commande. Les mots entre crochets sont facultatifs. Cela
signifie donc que toutes les options (commençant par des tirets) sont
facultatives, et que spécifier l'année est facultatif, de même que le mois si
l'année est présente, etc\ldots

\paragraph{} Par exemple, on peut écrire comme commandes:
\begin{itemize}
	\item \texttt{cal}, ce qui donne:
\begin{lstlisting}
   septembre 2015
lu ma me je ve sa di
    1  2  3  4  5  6
 7  8  9 10 11 12 13
14 15 16 17 18 19 20
21 22 23 24 25 26 27
28 29 30
\end{lstlisting}
	\item \texttt{cal -1}, qui nous donne le même résultat que la commande
		précédente
	\item \texttt{cal 2015}, qui nous donne le calendrier pour 2015
		(un peu long).
	\item \texttt{cal ----months 4 02 2014} (affiche 4 mois à partir de février
		2014), nous donnant donc:
\begin{lstlisting}[basicstyle=\footnotesize\ttfamily]
    février 2014            mars 2014            avril 2014
lu ma me je ve sa di  lu ma me je ve sa di  lu ma me je ve sa di
                1  2                  1  2      1  2  3  4  5  6
 3  4  5  6  7  8  9   3  4  5  6  7  8  9   7  8  9 10 11 12 13
10 11 12 13 14 15 16  10 11 12 13 14 15 16  14 15 16 17 18 19 20
17 18 19 20 21 22 23  17 18 19 20 21 22 23  21 22 23 24 25 26 27
24 25 26 27 28        24 25 26 27 28 29 30  28 29 30
                      31
      mai 2014
lu ma me je ve sa di
          1  2  3  4
 5  6  7  8  9 10 11
12 13 14 15 16 17 18
19 20 21 22 23 24 25
26 27 28 29 30 31
\end{lstlisting}
\end{itemize}


\section{Liste de commande étendue (intermédiaire)}
\begin{enumerate}
\item[make/gcc]
\item[du]
\item[df]
\item[touch]
\item[find]
\item[locate]
\item[chmod]
\item[chown]
\item[kill]
\item[nano]
\item[tar]
\item[wget]
\item[flux redirection]
\end{enumerate}
regexp simple, logiciel sympa (tmux, mocp, lua/bash/python),

\section{Regexp (expert)}
\begin{enumerate}
\item[mount]
\item[ncdu]
\item[halt/reboot]
\item[find]
\item[ps]
\item[awk]
\item[sed]
\item[ssh]
\item[git]
\item[usermod]
\item[groupmod]
\item[ifconfig]
\item[grep]
\end{enumerate}
un peu de réseaux? services,
dossier de base (/etc/, /var/, /home/, /mnt/)

\section{Une interface graphique c'est bien aussi}
lolwoot, y'a vraiment des gens qui font des apt-get install? 

\end{document}
